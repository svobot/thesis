\chapter{Linear Type Theory}\label{cha:linear}

So far, our type systems corresponded via the~Curry-Howard isomorphism to
the~traditional \emph{intuitionistic logic}. In this section we will consider
a~type system that is based on a~different logic. It is called \emph{linear
logic} and it was first described by Jean-Yves Girard in
1987~\cite{girard_1987}. In the~same way that the~intuitionistic logic is well
suited for reasoning about truth, the~linear logic is useful for reasoning about
resource usage. While until now we could freely duplicate, or discard
assumptions, in linear logic we give up this freedom. When we formulate the~type
system based on this logic, we will be able to reject expressions that take
an~overly cavalier approach to the~preservation of resources.

As an~example, we may want to reject the~deduction of judgments like
\begin{align}\label{eq:linear_examples}
  x \is{} \sigma \vdash (x, x) \is{} (\sigma \times \sigma)&  &  &\text{or}  &
    &y \is{} \sigma, z \is{} \tau \vdash z \is{} \tau,
\end{align}
because $x$ could represent an~extensive computer memory region and copying
the~data would be costly, or $y$ may represent a~persistent resource, like 
a~file handle, and we want to make sure that when we stop using the~file,
the~handle is released.

\sectionwithtoc{Structural properties}

Typing rules come in three kinds. First, we have \emph{axioms}, the~rules
without antecedents. Axiom is, for example:
\begin{mathpar}
  \inferrule*{ }{\vdash \univ \is{} \univ}.
\end{mathpar}

Second, there are the~\emph{logical} rules. These come in pairs of
\emph{introduction} and \emph{elimination} rules. For example, in
the~\nameref{sec:stlc}, the~introduction rule \ir{Lam} deduces a~judgment
assigning a~function type to a~newly formed term, while the~elimination rule
\ir{App} expects term of a~function type as one of its antecedents.

Third, there are three \emph{structural} rules, \emph{Exchange} \ir{Ex},
\emph{Contraction} \ir{Cont}, and \emph{Weakening} \ir{Weak}:
\begin{mathpar}
  \inferrule*[right=Ex]
  {\Gamma_1, \Gamma_2 \vdash M \is{} \sigma}
  {\Gamma_2, \Gamma_1 \vdash M \is{} \sigma}

  \inferrule*[right=Cont]
  {\Gamma, y \is{} \sigma, z \is{} \sigma \vdash M \is{} \tau}
  {\Gamma, x \is{} \sigma \vdash [x/y][x/z]M \is{} \tau}

  \inferrule*[right=Weak]
  {\Gamma \vdash M \is{} \tau}
  {\Gamma, x \is{} \sigma \vdash M \is{} \tau}
\end{mathpar}

The~rule \ir{Ex} expresses that the~order of assumptions is irrelevant,
\ir{Cont} expresses that any assumption may be duplicated, and \ir{Weak}
expresses that any assumption may be discarded. The~way we defined the~typing
systems in \autoref{cha:typesystems}, however, hides the~usage of these rules
and since the~structural rules are what differentiates the~intuitionistic and
linear logics, we will first reformulate the~intuitionistic rules such that
the~structural rules are emphasised.

The~rules \ir{Var}, \ir{App}, \ir{Lam}, \ir{Let}, and \ir{Pair}, defined in
the~\nameref{sec:stlc} section, are equivalent \todo{[PROOF?]} to the~following
rules, together with \ir{Ex}, \ir{Cont}, and \ir{Weak}. By equivalent we mean
that both sets of rules assign a~type to the~same set of terms.
\begin{mathpar}
  \inferrule*[right=Id]
  { }
  {x \is{} \sigma \vdash x \is{} \sigma} \\

  \inferrule*[right=$\to$-I]
  {\Gamma, x \is{} \sigma \vdash M \is{} \tau}
  {\Gamma \vdash \lam {x^\sigma} M \is{} \sigma \to \tau}

  \inferrule*[right=$\to$-E]
  {\Gamma_1 \vdash M \is{} \sigma \to \tau \\ \Gamma_2 \vdash N \is{} \sigma}
  {\Gamma_1, \Gamma_2 \vdash M \: N \is{} \tau}

  \inferrule*[right=$\times$-I]
  {\Gamma_1 \vdash M \is{} \sigma \\ \Gamma_2 \vdash N \is{} \tau}
  {\Gamma_1, \Gamma_2 \vdash \mpair M N \is{} \sigma \times \tau}

  \inferrule*[right=$\times$-E]
  {
    \Gamma_1 \vdash M \is{} \sigma \times \tau \\\\
    \Gamma_2, x \is{} \sigma, y \is{} \tau \vdash N \is{} \rho
  }
  {
    \Gamma_1, \Gamma_2 \vdash \textrm{let} \: \mpair x y = M \:
      \textrm{in} \: N \is{} \rho
  }
\end{mathpar}

The~first rule, \ir{Id}, is an~axiom and the~rest are the~logical rules, where
the~-I suffix in a~name stands for \emph{introduction} and -E for
\emph{elimination}.

\sectionwithtoc{Alternative pair rules}

Consider that instead of the~\emph{let} term in
\autoref{def:stlc}\ref{def:stlc:e_item}, which binds new variables to both
elements of a~pair at the~same time, we could introduce two new terms that would
each bind a~variable to only one element of a~pair. We can modify
the~definition such that:
\begin{enumerate}
  \setcounter{enumi}{\value{stlc_counter}}
  \item if $M^{\sigma \times \tau}$ is a~typed $\lambda$-term of type $\sigma
    \times \tau$, then the~following are typed $\lambda$-terms of types $\sigma$
    and $\tau$, respectively:
    \begin{mathpar}
      (\fst M^{\sigma \times \tau})^\sigma \and
      (\snd \: M^{\sigma \times \tau})^\tau.
    \end{mathpar}
\end{enumerate}

The~introduction and elimination rules \ir{$\times$-I} and \ir{$\times$-E} can
then be replaced by the~following:
\begin{mathpar}
  \inferrule*[right=$\times$-I$'$]
  {\Gamma \vdash M \is{} \sigma \\ \Gamma \vdash N \is{} \tau}
  {\Gamma \vdash \mpair M N \is{} \sigma \times \tau} \\

  \inferrule*[right=$\times$-E$'_1$]
  {\Gamma \vdash M \is{} \sigma \times \tau}
  {\Gamma \vdash \fst M \is{} \sigma}

  \inferrule*[right=$\times$-E$'_2$]
  {\Gamma \vdash M \is{} \sigma \times \tau}
  {\Gamma \vdash \snd M \is{} \tau}
\end{mathpar}

The~equivalence between the~old rules \ir{$\times$-I} and \ir{$\times$-E}, and
the~new \ir{$\times$-I$'$}, \ir{$\times$-E$'_1$}, and \ir{$\times$-E$'_2$} can
be established with a~few applications of the~structural
rules~\cite{wadler_1993}. The~terms can also be expressed using each other;
the~new \emph{\fst{}} and \emph{\snd{}} terms can be defined using \emph{let}:
\begin{align*}
  \fst M &:= \textrm{let} \: \mpair x y = M \: \textrm{in} \: x \\
  \snd M &:= \textrm{let} \: \mpair x y = M \: \textrm{in} \: y
\end{align*}
and, conversely, \emph{let} can be defined using \emph{\fst{}} and
\emph{\snd{}}:
\[
  \textrm{let} \: \mpair x y = M \: \textrm{in} \: N := (\lam x {\lam y N}) \:
    (\fst M) \: (\snd M).
\]

\sectionwithtoc{Substructural type system}

A~type system created with some of the~structural rules missing is called
a~\emph{substructural} type system. We can identify several type systems with
interesting properties by choosing which structural rules to omit from
the~system:
\begin{itemize}
  \item an~\emph{unrestricted} type system uses all three structural rules,
    Exchange, Contraction, and Weakening;
  \item an~\emph{affine} type system omits the~Contraction rule and uses only
    the~Exchange and Weakening rules;
  \item a~\emph{relevant} type system omits the~Weakening rule and uses only
    the~Exchange and Contraction rules;
  \item a~\emph{linear} type system omits both the~Contraction and Weakening
    rules and uses only the~Exchange rule;
  \item an~\emph{ordered} type system omits all three structural rules.
\end{itemize}
We will focus on a~linear type system, but rather then getting rid of
Contraction and Weakening outright, we want to formulate a~type system, that
still lets us embed within it the~unrestricted type system we developed in
the~\nameref{sec:stlc} section, while also giving us the~control over when
Contraction and Weakening can be used.

Recall the~problematic judgments (\ref{eq:linear_examples}) we saw at
the~beginning of this chapter and notice how the~deduction trees make use of
the~structural rules:
\begin{mathpar}
  \inferrule*[Right=Cont]
  {
    \inferrule*[Right=$\times$-I]
    {
      \inferrule*[right=Id]
      { }
      {x \is{} \sigma \vdash x \is{} \sigma} \\
      \inferrule*[Right=Id]
      { }
      {x \is{} \sigma \vdash x \is{} \sigma}
    }
    {
      x \is{} \sigma, x \is{} \sigma \vdash (x, x) \is{} (\sigma \times \sigma)
    }
  }
  {x \is{} \sigma \vdash (x, x) \is{} (\sigma \times \sigma)}

  \inferrule*[Right=Weak]
  {
    \inferrule*[Right=Id]
    { }
    {z \is{} \tau \vdash z \is{} \tau}
  }
  {y \is{} \sigma, z \is{} \tau \vdash z \is{} \tau}
\end{mathpar}
We can now see that without the~\ir{Cont} and \ir{Weak} rules, we would be
unable to deduce these judgments.

Similarly, with the~\ir{Cont} and \ir{Weak} rules missing, we cannot establish
the~equivalence between the~two distinct ways of defining the~pair rules; thus,
in linear type systems, \ir{$\times$-I} and \ir{$\times$-E} define a~different
kind of pair from \ir{$\times$-I$'$}, \ir{$\times$-E$'_1$}, and
\ir{$\times$-E$'_2$}. To distinguish between these, we will use two different
conectives, $\otimes$ for the~former pair definition and $\with$ for the~latter.

While there are some works that build linear type systems with dependent
types~\cite{cervesato_pfenning_2002, krishnaswami_et_al_2015}, we will first
present a~linear type system with simple types and only in \autoref{cha:qtt},
combine the~notions of linear and dependent types.

\sectionwithtoc{Linear types}

Linear types are an~extension of the~simple types in \autoref{def:simple_type}.

\begin{definition}
  Assume that we are given a~finite or infinite sequence of symbols called
  \emph{atomic types}; then we define \emph{linear types} inductively: every
  atomic type is a~linear type and if $\sigma$ and $\tau$ are linear types, then
  the~following are also linear types:
  \begin{enumerate}
    \item a~\emph{linear function type} $\sigma \multimap \tau$;
    \item a~\emph{multiplicative pair type} $\sigma \otimes \tau$;
    \item an~\emph{additive pair type} $\sigma \with \tau$;
    \item an~\emph{exponential type} $\oc \sigma$;
  \end{enumerate}
\end{definition}

Without Contraction and Weakening the~type connectives have very different
meaning from their simple-type analogues. A~linear function $\sigma \multimap
\tau$ can be read as ``consume $\sigma$ yielding $\tau$.'' As noted previously,
there are now two distinct ways of defining a~pair type; these are written
$\sigma \otimes \tau$, pronounced ``both $\sigma$ and $\tau$,'' and $\sigma
\with \tau$, pronounced ``choose from $\sigma$ and $\tau$.'' A~new type
$\oc \sigma$, pronounced ``of course $\sigma$,'' is used to indicate whether
Contraction or Weakening may be used.

\sectionwithtoc{Linear terms \& contexts}

Our particular linear type system will be based on Girard's Logic of
Unity~\cite{girard_1993}, which is a~refinement of linear logic. It simplifies
some things by using two forms of assumptions; the~first form, called
\emph{intuitionistic}, allows Contraction and Weakening, but the~second form,
called \emph{linear}, does not. We will split the~context of our typing
judgments into two parts, separated by a~vertical bar, with intuitionistic
assumptions in the~first part and linear in the~second:
\[
  x_1 \is{} \sigma_1, \dotsc, x_n \is{} \sigma_n \mid y_1 \is{} \tau_1, \dotsc,
    y_m \is{} \tau_m \vdash M \is{} \rho.
\]

Like before, $\diamond$ denotes an~empty list of assumptions and $\Gamma,
\Delta$ range over lists of zero or more assumptions.

There is a~way to define a~sound system of linear logic that does not use
intuitionistic assumptions, such as the~one described by Benton
\textit{et~al.}~\cite{benton_et_al_1993}, but the~resulting system has
considerably more complex deduction rules.

Like with the~simply typed \lc, the~\lc with linear types has an~inductive
definition of terms and a~set of judgment rules that assign types to these
terms.

\begin{definition}
  The~set of all terms of a~linear type system is defined inductively:
  \begin{enumerate}
    \item all typed variables $x^\tau$ are terms of type $\tau$;
    \item if $M^{\sigma \multimap \tau}$ and $N^\sigma$ are terms of types
      $\sigma \multimap \tau$ and $\sigma$ respectively, then the~following is
      a~term of type $\tau$:
      \[
        (M^{\sigma \multimap \tau} \: N^\sigma)^\tau;
      \]
    \item if $x^\sigma$ is a~typed variable and $M^\tau$ is a~term of type
      $\tau$, then the~following is a~term of type $\sigma \multimap \tau$:
      \[
        (\lam {x^\sigma} {M^\tau})^{\sigma \multimap \tau};
      \]
    \item if $M^\sigma$ and $N^\tau$ are terms of types $\sigma$ and $\tau$
      respectively, then the~following is a~term of type $\sigma \otimes
      \tau$:
      \[
        \mpair {M^\sigma} {N^\tau} ^{\sigma \otimes \tau};
      \]
    \item if $x^\sigma$ and $y^\tau$ are typed variables and $M^{\sigma \otimes
      \tau}$ and $N^\rho$ are terms of type $\sigma \otimes \tau$ and $\rho$
      respectively, then the~following is a~term of type $\rho$:
      \[
        (\textrm{let} \: \mpair {x^\sigma} {y^\tau} = M^{\sigma \otimes \tau} \:
        \textrm{in} \: N^{\rho})^\rho;
      \]
    \item if $M^\sigma$ and $N^\tau$ are terms of types $\sigma$ and $\tau$
      respectively, then the~following is a~term of type $\sigma \with
      \tau$:
      \[
        \apair {M^\sigma} {N^\tau} ^{\sigma \with \tau};
      \]
    \item if $M^{\sigma \with \tau}$ is a~term of type $\sigma \with \tau$, then
      the~following are terms of type $\sigma$ and $\tau$ respectively:
      \begin{align*}
        (\fst M^{\sigma \with \tau})^\sigma&  &
          &(\snd M^{\sigma \with \tau})^\tau;
      \end{align*}
    \item if $M^\sigma$ is a~term of type $\sigma$, then the~following is a~term
      of type $\oc \sigma$:
      \[
        (\oc M)^{\oc \sigma}.
      \]
  \end{enumerate}
\end{definition}

\sectionwithtoc{Linear typing rules}

Except for the~separation of intuitionistic and linear assumptions into two
parts, the~typing rules for $\multimap, \otimes$, and $\with$ are identical to
the~typing rules for $\to, \times$, and the~alternative rules for $\times$,
respectively.

The~axioms and the~structural rules are the~same; they just work on the~two-part
assumptions list.
\begin{mathpar}
  \inferrule*[right=Id$_1$]
  { }
  {x \is{} \sigma \mid \diamond \vdash x \is{} \sigma}

  \inferrule*[right=Id$_2$]
  { }
  {\Gamma \mid x \is{} \sigma \vdash x \is{} \sigma}

  \inferrule*[right=Ex]
  {\Gamma_1, \Gamma_2 \mid \Delta_1, \Delta_2 \vdash M \is{} \sigma}
  {\Gamma_2, \Gamma_1 \mid \Delta_2, \Delta_1 \vdash M \is{} \sigma}

  \inferrule*[right=Cont]
  {\Gamma, y \is{} \sigma, z \is{} \sigma \mid \Delta \vdash M \is{} \tau}
  {\Gamma, x \is{} \sigma \mid \Delta \vdash [x/y][x/z]M \is{} \tau}

  \inferrule*[right=Weak]
  {\Gamma \mid \Delta \vdash M \is{} \tau}
  {\Gamma, x \is{} \sigma \mid \Delta \vdash M \is{} \tau}
\end{mathpar}
With Contraction and Weakening available in the~intuitionistic part of
the~context, the~intuitionistic assumption $x \is{} \sigma$ can be thought of as
supplying an~unlimited number of occurrences of $\sigma$; with Contraction used,
we can supply multiple occurrences of $\sigma$, and with Weakening, we can
supply no occurrences. The~linear assumption $x \is{} \sigma$, on the~other
hand, supplies exactly one occurrence of $\sigma$.

In the~introduction rule for linear functions, the~binding variable referencing
the~function parameter forms a~linear hypothesis in the~premise; hence,
the~function argument must be used in $M$. The~elimination rule states that
the~function application must consume the~same resources as both the~function
and its argument put together.
\begin{mathpar}
  \inferrule*[right=$\multimap$-I]
  {\Gamma \mid \Delta, x \is{} \sigma \vdash M \is{} \tau}
  {\Gamma \mid \Delta \vdash \lam {x^\sigma} M \is{} \sigma \multimap \tau}

  \inferrule*[right=$\multimap$-E]
  {
    \Gamma \mid \Delta_1 \vdash M \is{} \sigma \multimap \tau \\\\
    \Gamma \mid \Delta_2 \vdash N \is{} \sigma
  }
  {\Gamma \mid \Delta_1, \Delta_2 \vdash M \: N \is{} \tau}
\end{mathpar}

In the~rules for terms of multiplicative pair types, the~introduction rule is
similar to function application; the~resources consumed by the~derived term
equal the~resources consumed by the~term's constituents. The~elimination rule
deconstructs the~pair via the~\emph{let} term and binds both components to new
variables. The~linearity of the~two new hypotheses means that the~variables must
both be used in $N$.
\begin{mathpar}
  \inferrule*[right=$\otimes$-I]
  {
    \Gamma \mid \Delta_1 \vdash M \is{} \sigma \\\\
    \Gamma \mid \Delta_2 \vdash N \is{} \tau
  }
  {
    \Gamma \mid \Delta_1, \Delta_2 \vdash \mpair M N \is{} \sigma \otimes \tau
  }

  \inferrule*[right=$\otimes$-E]
  {
    \Gamma \mid \Delta_1 \vdash M \is{} \sigma \otimes \tau \\\\
    \Gamma \mid \Delta_2, x \is{} \sigma, y \is{} \tau \vdash N \is{} \rho
  }
  {
    \Gamma \mid \Delta_1, \Delta_2
    \vdash \textrm{let} \: \mpair x y = M \: \textrm{in} \: N \is{} \rho
  }
\end{mathpar}

In the~additive pairs, both elements have to use all the~linear resources and
the~elimination rules ensure that only one of the~elements can be extracted.

\begin{mathpar}
  \inferrule*[right=$\with$-I]
  {
    \Gamma \mid \Delta \vdash M \is{} \sigma \\
    \Gamma \mid \Delta \vdash N \is{} \tau
  }
  {\Gamma \mid \Delta \vdash \apair M N \is{} \sigma \with \tau} \\

  \inferrule*[right=$\with$-E$_1$]
  {\Gamma \mid \Delta \vdash M \is{} \sigma \with \tau}
  {\Gamma \mid \Delta \vdash \fst M \is{} \sigma}

  \inferrule*[right=$\with$-E$_2$]
  {\Gamma \mid \Delta \vdash M \is{} \sigma \with \tau}
  {\Gamma \mid \Delta \vdash \snd M \is{} \tau}
\end{mathpar}

Before we define the~remaining typing rules of the~system, the~two logical rules
that involve the~exponential type, we will examine the~differences between
the~additive and multiplicative pair types, quoting Wadler's
example~\cite{wadler_1993}. Under the~Curry-Howard correspondence, take $\sigma$
to be the~proposition ``I have 10 euro,'' $\tau$ to be the~proposition ``I have
a~pizza,'' and $\rho$ to be the~proposition ``I have a~cake.'' The~judgments
\begin{mathpar}
  \diamond \mid x \is{} \sigma \vdash M \is{} \tau \and
  \diamond \mid x \is{} \sigma \vdash N \is{} \rho
\end{mathpar}
express that for 10 euro we may buy a~pizza, and for 10 euro we may buy a~cake.
With the~rule \ir{$\otimes$-I}, we can derive
\begin{mathpar}
  \inferrule*[right=$\otimes$-I]
  {
    \inferrule*
    {}
    {\diamond \mid x_1 \is{} \sigma \vdash M \is{} \tau} \\
    \inferrule*
    {}
    {\diamond \mid x_2 \is{} \sigma \vdash N \is{} \rho}
  }
  {
    \diamond \mid x_1 \is{} \sigma, x_2 \is{} \sigma
    \vdash \mpair M N \is{} \tau \otimes \rho
  }
\end{mathpar}
meaning that for 20 euro we can buy both a~pizza and a~cake. With the~rule
\ir{$\with$-I} we can derive
\begin{mathpar}
  \inferrule*[right=$\with$-I]
  {
    \inferrule*
    {}
    {\diamond \mid x \is{} \sigma \vdash M \is{} \tau} \\
    \inferrule*
    {}
    {\diamond \mid x \is{} \sigma \vdash N \is{} \rho}
  }
  {
    \diamond \mid x \is{} \sigma \vdash \apair M N \is{} \tau \with \rho
  }
\end{mathpar}
meaning that for 10 euro we can buy whichever we choose from a~pizza or a~cake.

Note that $\tau \with \rho$ represents our \emph{choice} of either pizza or
cake, which is different from \emph{having} one of pizza or cake. To say that we
have either a~pizza or a~cake, we would use the~sum type $\tau \oplus \rho$. We
have not formally presented the~sum type $\tau \oplus \rho$ before, because we
shall not use it outside of this example and we mention it here only to point
out the~difference from $\tau \with \rho$. With the~sum type $\tau \oplus \rho$
we can derive
\begin{mathpar}
  \inferrule*[right=$\oplus$-I$_1$]
  {\diamond \mid x \is{} \sigma \vdash M \is{} \tau}
  {\diamond \mid x \is{} \sigma \vdash M \is{} \tau \oplus \rho}

  \inferrule*[right=$\oplus$-I$_2$]
  {\diamond \mid x \is{} \sigma \vdash N \is{} \rho}
  {\diamond \mid x \is{} \sigma \vdash N \is{} \tau \oplus \rho}
\end{mathpar}
The first derivation means that if 10 euro buys us a~pizza, then it also buys us
a~pizza or a~cake. The~second derivation is analogous, starting with a~cake.

Now consider the~implications of having a~pair as a~hypothesis. Take $\phi$ to
be the~proposition ``I am happy.'' Then the~judgments
\begin{mathpar}
  \diamond \mid x \is{} \tau \otimes \rho \vdash M \is{} \phi \and
  \diamond \mid x \is{} \tau \with \rho \vdash M \is{} \phi \and
  \diamond \mid x \is{} \tau \oplus \rho \vdash M \is{} \phi
\end{mathpar}
mean that, first, I will be happy given \emph{both} a~pizza and a~cake; second,
I will be happy given \emph{my choice} from a~pizza and a~cake; and, third, I
will be happy given \emph{either} a~pizza or a~cake, I do not care which.

With this intuition, we can examine how the~pair types relate to each other.
Neither of the~following judgments is provable:
\begin{mathpar}
  \diamond \mid x \is{} \tau \otimes \rho \vdash M \is{} \tau \with \rho \and
  \diamond \mid x \is{} \tau \with \rho \vdash M \is{} \tau \otimes \rho
\end{mathpar}
Having a~pizza and a~cake, we cannot choose to have one of pizza or cake,
because that would mean we threw away the~other thing, which is not admissible
without the~\ir{Weak} rule~\todo{[CORRECT?]}. On the~other hand, having
the~choice of either a~pizza or a~cake does not mean we have both.

There are two proofs of $\diamond \mid x \is{} \tau \with \rho \vdash M \is{}
\tau \oplus \rho$:
\begin{mathpar}
  \inferrule*
  {
    \inferrule*[Right=$\with$-E$_1$]
    {
      \inferrule*[Right=Id$_2$]
      { }
      {\diamond \mid x \is{} \tau \with \rho \vdash x \is{} \tau \with \rho}
    }
    {\diamond \mid x \is{} \tau \with \rho \vdash \fst x \is{} \tau}
  }
  {\diamond \mid x \is{} \tau \with \rho \vdash \fst x \is{} \tau \oplus \rho}

  \inferrule*
  {
    \inferrule*[Right=$\with$-E$_2$]
    {
      \inferrule*[Right=Id$_2$]
      { }
      {\diamond \mid x \is{} \tau \with \rho \vdash x \is{} \tau \with \rho}
    }
    {\diamond \mid x \is{} \tau \with \rho \vdash \snd x \is{} \rho}
  }
  {\diamond \mid x \is{} \tau \with \rho \vdash \snd x \is{} \tau \oplus \rho}
\end{mathpar}
We can either choose the~pizza, or the~cake. There is, however, no proof of
the~converse, $\diamond \mid x \is{} \tau \oplus \rho \vdash M \is{} \tau \with
\rho$; having a~pizza or a~cake does not constitute having the~\emph{choice} of
a~pizza or a~cake.

Now we return to the~remaining typing rules of our system. They are
the~introduction and elimination rules of the~exponential type. Notice that, in
this system, the~logical rules have so far all been defined in terms of linear
assumptions. To make use of our intuitionistic assumptions, we need to turn them
into linear ones, which is done using the~\oc{} connective; in fact,
an~intuitionistic assumption $x \is{} \sigma$ is in a~sense equivalent to linear
$\oc x \is{} \oc \sigma$.

The~introduction rule is applicable when the~context contains only
intuitionistic assumptions and it states that if we can prove $M \is{} \sigma$
in this context, we can also prove $\oc M \is{} \oc \sigma$ in the same context;
meaning, if we can derive a~single instance of $\sigma$ from the~unlimited
resources $\Gamma$, we can also derive an~unlimited number of instances of
$\sigma$.

The~elimination rule states that we can use an~instance of type $\oc \sigma$ to
satisfy the~intuitionistic assumption $x \is{} \sigma$.
\begin{mathpar}
  \inferrule*[right=$\oc{}$-I]
  {\Gamma \mid \diamond \vdash M \is{} \sigma}
  {\Gamma \mid \diamond \vdash \oc M \is{} \oc \sigma}

  \inferrule*[right=$\oc{}$-E]
  {
    \Gamma \mid \Delta_1 \vdash M \is{} \oc \sigma \\
    \Gamma, x \is{} \sigma \mid \Delta_2 \vdash N \is{} \tau
  }
  {
    \Gamma \mid \Delta_1, \Delta_2
    \vdash \textrm{let} \: \oc x = M \: \textrm{in} \: N \is{} \tau
  }
\end{mathpar}

The~difference between intuitionistic assumption $x \is{} \sigma$ and linear
assumption $\oc x \is{} \oc \sigma$ is that the~linear assumption still must be
used exactly once in the~derivation tree, while the~intuitionistic one may be
used any number of times. They are equivalent, however, in the~sense that
$\Gamma \mid \Delta, x \is{} \oc \sigma \vdash M \is{} \tau$ is provable if
and only if $\Gamma, x \is{} \sigma \mid \Delta \vdash N \is{} \tau$ is
provable. This is shown by the~following derivation trees:
\begin{mathpar}
  \inferrule*[Right=$\multimap$-E]
  {
    \inferrule*[right=Weak]
    {
      \inferrule*[Right=$\multimap$-I]
      {\Gamma \mid \Delta, x \is{} \oc \sigma \vdash M \is{} \tau}
      {
        \Gamma \mid \Delta
        \vdash \lam {x^{\oc \sigma}} M \is{} \oc \sigma \multimap \tau
      }
    }
    {
      \Gamma, x \is{} \sigma \mid \Delta
      \vdash \lam {x^{\oc \sigma}} M \is{} \oc \sigma \multimap \tau
    } \\
    \inferrule*[Right=$\oc{}$-I]
    {
      \inferrule*[Right=Weak]
      {
        \inferrule*[Right=Weak]
        {
          \inferrule*[Right=Id$_1$]
          { }
          {x \is{} \sigma \mid \diamond \vdash x \is{} \sigma}
        }
        {\vdots}
      }
      {\Gamma, x \is{} \sigma \mid \diamond \vdash x \is{} \sigma}
    }
    {\Gamma, x \is{} \sigma \mid \diamond \vdash \oc x \is{} \oc \sigma}
  }
  {
    \Gamma, x \is{} \sigma \mid \Delta
    \vdash (\lam {x^{\oc \sigma}} M) \: \oc x \is{} \tau
  }
\end{mathpar}

\begin{mathpar}
  \inferrule*[Right=$\oc{}$-E]
  {
    \inferrule*[right=Id$_2$]
    { }
    {\diamond \mid x \is{} \oc \sigma \vdash x \is{} \oc \sigma} \\
    \Gamma, x \is{} \sigma \mid \Delta \vdash N \is{} \tau
  }
  {
    \Gamma \mid \Delta, x \is{} \oc \sigma
    \vdash \textrm{let} \: \oc x = x \: \textrm{in} \: N \is{} \tau
  }
\end{mathpar}

The~reason why we indeed need separate intuitionistic assumptions $x \is{}
\sigma$ and cannot use $\oc x \is{} \oc \sigma$ instead is explained by
Wadler~\cite{wadler_1993}.

