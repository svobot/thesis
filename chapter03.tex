\chapter{Linear Type System}

So far, our type systems were based on the~traditional \emph{intuitionistic}
logic via the~\emph{Curry-Howard isomorphism}, which views propositions of logic
as types in a~program. Under this correspondence, proof of a~proposition is
a~correctly-typed term in a~given calculus. In this section we will consider
a~type system that is based on a~different logic. It is called \emph{linear
logic} and it was first described by Jean-Yves Girard in
1987~\cite{girard_1987}. In the~same way that the~intuitionistic logic is well
suited for reasoning about truth, the~linear logic is useful for reasoning about
resource usage. While until now we could freely duplicate, or discard
assumptions, in linear logic we give up this freedom. When we formulate the~type
system based on this logic, we will be able to reject expressions that take
an~overly cavalier approach to the~preservation of resources.

As an~example, we may want to reject the~deduction of judgments like
\begin{align}\label{eq:linear_examples}
  x \is{} S \vdash (x, x) \is{} (S \times S)&  &  &\text{or}  &
    &y \is{} S, z \is{} T \vdash z \is{} T,
\end{align}
because $x$ could represent an~extensive computer memory region and copying
the~data would be costly, or $y$ may represent a~persistent resource, like 
a~file handle, and we want to make sure that when we stop using the~file,
the~handle is released.

\sectionwithtoc{Structural properties}

Typing rules come in three kinds. First, we have \emph{axioms}, the~rules
without antecedents. Axiom is, for example:
\begin{mathpar}
  \inferrule*{ }{\vdash \univ \is{} \univ}.
\end{mathpar}

Second, there are the~\emph{logical} rules. These come in pairs of
\emph{introduction} and \emph{elimination} rules. For example, in the~simply
typed \lc, the~introduction rule \ir{Lam} deduces a~judgment assigning
a~function type to a~newly formed term, while the~elimination rule \ir{App}
expects term of a~function type as one of its antecedents.

Third, there are three \emph{structural} rules, \emph{exchange},
\emph{contraction}, and \emph{weakening}. The~way we defined the~typing systems
in \autoref{cha:typesystems}, however, hides the~usage of these rules and since
the~structural rules are what differentiates the~intuitionistic and linear
logics, we will first reformulate the~intuitionistic rules such that
the~structural rules are emphasised.

The~following rules are equivalent \todo{[PROOF?]} to the~ones we defined in
the~\nameref{sec:stlc} section:
\begin{mathpar}
  \inferrule*[right=Id]
  { }
  {x \is{} S \vdash x \is{} S}

  \inferrule*[right=Ex]
  {\Gamma_1, \Gamma_2 \vdash M \is{} S}
  {\Gamma_2, \Gamma_1 \vdash M \is{} S}

  \inferrule*[right=Cont]
  {\Gamma, y \is{} S, z \is{} S \vdash M \is{} T}
  {\Gamma, x \is{} S \vdash [x/y][x/z]M \is{} T}

  \inferrule*[right=Weak]
  {\Gamma \vdash M \is{} T}
  {\Gamma, x \is{} S \vdash M \is{} T}

  \inferrule*[right=$\to$-I]
  {\Gamma, x \is{} S \vdash M \is{} T}
  {\Gamma \vdash \lam x M \is{} S \to T}

  \inferrule*[right=$\to$-E]
  {\Gamma_1 \vdash M \is{} S \to T \\ \Gamma_2 \vdash N \is{} S}
  {\Gamma_1, \Gamma_2 \vdash M \: N \is{} T}

  \inferrule*[right=$\times$-I]
  {\Gamma_1 \vdash M \is{} S \\ \Gamma_2 \vdash N \is{} T}
  {\Gamma_1, \Gamma_2 \vdash \mpair M N \is{} S \times T}

  \inferrule*[right=$\times$-E]
  {
    \Gamma_1 \vdash M \is{} S \times T \\
    \Gamma_2, x \is{} S, y \is{} T \vdash N \is{} U
  }
  {
    \Gamma_1, \Gamma_2 \vdash \textrm{let} \: \mpair x y = M \:
      \textrm{in} \: N \is{} U
  }
\end{mathpar}

In this set of rules, the~first, \ir{Id}, is an~axiom; the~following three are
the~structural rules, exchange \ir{Ex}, contraction \ir{Cont}, and weakening
\ir{Weak}; and the~rest are the~logical rules, where the~-I suffix in a~name of
a~rule stands for \emph{introduction} and -E for \emph{elimination}.

The~\ir{Ex} rule expresses that the order of assumptions is irrelevant,
\ir{Cont} expresses that any assumption may be duplicated, and \ir{Weak}
expresses that any assumption may be discarded.

\sectionwithtoc{Alternative pair rules}

Consider that instead of the~\emph{let} term in
\autoref{def:stlc}\ref{def:stlc:e_item}, which binds new variables to both
elements of a~pair at the~same time, we could introduce two new terms that would
each bind a~variable to only one element of a~pair. We can modify
the~\autoref{def:stlc} such that:
\begin{enumerate}
  \setcounter{enumi}{\value{stlc_counter}}
  \item if $M^{\sigma \times \tau}$ is typed $\lambda$-term of type $\sigma
    \times \tau$, then the~following typed $\lambda$-terms have types $\sigma$
    and $\tau$, respectively:
    \begin{align*}
      (\pi_1 \: M^{\sigma \times \tau})^\sigma&  &
        &(\pi_2 \: M^{\sigma \times \tau})^\tau.
    \end{align*}
\end{enumerate}

The~introduction and elimination rules for pair values can also be expressed in
an~alternative form:
\begin{mathpar}
  \inferrule*[right=$\times$-I$'$]
  {\Gamma \vdash M \is{} S \\ \Gamma \vdash N \is{} T}
  {\Gamma \vdash \mpair M N \is{} S \times T} \\

  \inferrule*[right=$\times$-E$'_1$]
  {\Gamma \vdash M \is{} S \times T}
  {\Gamma \vdash \pi_1 \: M \is{} S}

  \inferrule*[right=$\times$-E$'_2$]
  {\Gamma \vdash M \is{} S \times T}
  {\Gamma \vdash \pi_2 \: M \is{} T}
\end{mathpar}

We can establish equivalence between these rules and the~old \ir{$\times$-I}
and \ir{$\times$-E} with a~few applications of the~structural
rules~\cite{wadler_1993}. Hence, the~terms can also be expressed using each
other; the~new $\pi_1$ and $\pi_2$ can be defined using the~\emph{let} term:
\begin{align*}
  \pi_1 \: M &:= \textrm{let} \: \mpair x y = M \: \textrm{in} \: x \\
  \pi_2 \: M &:= \textrm{let} \: \mpair x y = M \: \textrm{in} \: y,
\end{align*}
and conversely \emph{let} can be defined using $\pi_1$ and $\pi_2$:
\[
  \textrm{let} \: \mpair x y = M \: \textrm{in} \: N := (\lam x {\lam y N}) \:
    (\pi_1 \: M) \: (\pi_2 \: M).
\]

