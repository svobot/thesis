\chapter{Linear Type Theory}\label{cha:linear}

\coloredlettrine{O}{ur type} systems have so far corresponded via
the~Curry-Howard isomorphism to the~traditional \emph{intuitionistic logic}. In
this section we will consider a~type system that is based on a~different logic.
It is called \emph{linear logic} and it was first described by
\citet{girard_1987}. In the~same way that the~intuitionistic logic is well
suited for reasoning about truth, the~linear logic is useful for reasoning about
resource usage. While until now we could freely duplicate, or discard
assumptions, in linear logic we give up this freedom. When we formulate the~type
system based on this logic, we will be able to reject expressions that take
an~overly cavalier approach to the~preservation of resources.

As an~example, we may want to reject the~deduction of judgments like:
\begin{equation}\label{eq:linear_examples}
  x \is{} S \vdash (x, x) \is{} (S \times S) \qquad\qquad
    y \is{} S, z \is{} T \vdash z \is{} T,
\end{equation}
because $x$ could represent an~extensive computer memory region and copying
the~data would be costly, or $y$ may represent a~persistent resource, like 
a~file handle, and we want to make sure that when we stop using the~file,
the~handle is released.

\sectionwithtoc{Structural properties}

Typing rules come in three kinds. First, we have \emph{axioms}, the~rules
without antecedents.

Second, there are the~\emph{logical} rules. These come in pairs of
\emph{introduction} and \emph{elimination} rules as we have seen thorough
the~last chapter. For example, the~introduction rule \ir{$\to$-I} deduces
a~judgment assigning a~function type to a~newly formed term, while
the~elimination rule \ir{$\to$-E} expects term of a~function type as one of its
antecedents.

Third, there are three \emph{structural} rules, \emph{Exchange} \ir{Ex},
\emph{Contraction} \ir{Cont}, and \emph{Weakening} \ir{Weak}:
\begin{mathpar}
  \inferrule*[right=Ex]
  {\Gamma_1, \Gamma_2 \vdash M \is{} S}
  {\Gamma_2, \Gamma_1 \vdash M \is{} S}

  \inferrule*[right=Cont]
  {\Gamma, y \is{} S, z \is{} S \vdash M \is{} T}
  {\Gamma, x \is{} S \vdash M[x/y][x/z] \is{} T}

  \inferrule*[right=Weak]
  {\Gamma \vdash M \is{} T}
  {\Gamma, x \is{} S \vdash M \is{} T}
\end{mathpar}

The~rule \ir{Ex} expresses that the~order of assumptions is irrelevant,
\ir{Cont} expresses that any assumption may be duplicated, and \ir{Weak}
expresses that any assumption may be discarded. The~way we defined the~typing
systems in \autoref{cha:typesystems}, however, hides the~usage of these rules
and since the~structural rules are what differentiates the~intuitionistic and
linear logics, we will first reformulate the~intuitionistic rules such that
the~structural rules are emphasised.

The~new rules \ir{$\to$-I}, \ir{$\to$-E}, \ir{$\times$-I}, \ir{$\times$-E},
\ir{\1-I}, and \ir{\1-E}, which we define below, are, together with \ir{Id},
\ir{Ex}, \ir{Cont}, and \ir{Weak}, equivalent to the~old rules \ir{Var},
\ir{$\to$-I}, \ir{$\to$-E}, \ir{$\times$-I}, \ir{$\times$-E}, \ir{\1-I}, and
\ir{\1-E} defined in the~\nameref{sec:stlc} section. By equivalent we mean that
both sets of rules assign the~same type to every term. The~first rule, \ir{Id},
is an~axiom and the~rest are the~logical rules:
\begin{mathpar}
  \inferrule*[right=Id]
  { }
  {x \is{} S \vdash x \is{} S} \\

  \inferrule*[right=$\to$-I]
  {\Gamma, x \is{} S \vdash M \is{} T}
  {\Gamma \vdash (\lam {x \is{} S} M) \is{} S \to T}

  \inferrule*[right=$\to$-E]
  {\Gamma_1 \vdash M \is{} S \to T \\ \Gamma_2 \vdash N \is{} S}
  {\Gamma_1, \Gamma_2 \vdash M \: N \is{} T}

  \inferrule*[right=$\times$-I]
  {\Gamma_1 \vdash M \is{} S \\ \Gamma_2 \vdash N \is{} T}
  {\Gamma_1, \Gamma_2 \vdash \mpair M N \is{} S \times T}

  \inferrule*[right=$\times$-E]
  {
    \Gamma_1 \vdash M \is{} S \times T \\
    \Gamma_2, x \is{} S, y \is{} T \vdash N \is{} U
  }
  {
    \Gamma_1, \Gamma_2 \vdash \letpair{S \times T} x y M N \is{} U
  }

  \inferrule*[right=\1-I]
  { }
  {\Gamma \vdash \munit \is{} \1}

  \inferrule*[right=\1-E]
  {
    \Gamma_1 \vdash M \is{} \1 \\
    \Gamma_2 \vdash N \is{} S
  }
  {
    \Gamma_1, \Gamma_2 \vdash \letu M N \is{} S
  }
\end{mathpar}

\sectionwithtoc{Alternative pairs \& units}

Pairs of constructors and eliminators always have two ways of defining them. In
this section, we will go over the~alternative definitions of constructors and
eliminators for the~pair and unit types.

We will use a~different notation to distinguish the~alternative definitions;
the~pair constructor uses angle brackets and there are now two eliminators:
\begin{mathpar}
  \apair M N \and \fst M \and \snd M
\end{mathpar}

The~\emph{let} eliminator in \autoref{def:stlc} \ref{def:stlc:let_item}
binds new variables to both elements of the~pair at the~same time and it
β-reduces to the~\emph{in} sub-term, with both elements of the~pair in context.
The~two new eliminators, on the~other hand, each access only one of the~elements
and β-reduce as follows:
\begin{align*}
  \fst {\apair M N} \quad &\triangleright_{1\beta} \quad M \\
  \snd {\apair M N} \quad &\triangleright_{1\beta} \quad N
\end{align*}

The~introduction and elimination rules for the~new definition of pairs are:
\begin{mathpar}
  \inferrule*[right=$\times$-I$'$]
  {\Gamma \vdash M \is{} S \\ \Gamma \vdash N \is{} T}
  {\Gamma \vdash \apair M N \is{} S \times T} \\

  \inferrule*[right=$\times$-E$'_1$]
  {\Gamma \vdash M \is{} S \times T}
  {\Gamma \vdash \fst M \is{} S}

  \inferrule*[right=$\times$-E$'_2$]
  {\Gamma \vdash M \is{} S \times T}
  {\Gamma \vdash \snd M \is{} T}
\end{mathpar}

As both definitions of pairs form the~same type, the~terms and derivation rules
can be expressed using each other. The~equivalence between the~old rules
\ir{$\times$-I} and \ir{$\times$-E}, and the~new \ir{$\times$-I$'$},
\ir{$\times$-E$'_1$}, and \ir{$\times$-E$'_2$} can be established with a~few
applications of the~structural rules~\citep{wadler_1993}. The~equivalence of
constructors is trivial and the~eliminators are defined in terms of each other
like this: the~new \emph{fst} and \emph{snd} defined using \emph{let}:
\begin{align*}
  \fst M &\defeq \letpair {S \times T} x y M x \\
  \snd M &\defeq \letpair {S \times T} x y M y
\end{align*}
and, conversely, \emph{let} defined using \emph{fst} and \emph{snd}:
\[
  \letpair {S \times T} x y M N \defeq (\lam {x \is{} S} {\lam {y \is{} T} N})
  \: (\fst M) \: (\snd M)
\]

With unit types, our alternative definition uses the~notation \aunit for
the~constructor, but it has no eliminator. The~analogy with pairs works like
this: pairs have two elements so the~alternative definition of pair type has two
eliminators, then, since unit has zero elements, it also has zero eliminators.

The~introduction rule is identical to the~old definition, but now there is no
elimination rule:
\begin{mathpar}
  \inferrule*[right=\1-I$'$]
  { }
  {\Gamma \vdash \aunit \is{} \1}
  \and
  \text{\emph{No elimination rule}}
\end{mathpar}
Without the~elimination rule we can never do anything interesting with a~value
of this type, except to incorporate it into a~larger value.

We can establish the~equivalence on terms with the~following: every occurrence
of the~unit eliminator can be expressed using only the~\emph{in} part of
the~term:
\[
  \letu M N \defeq N
\]
The~opposite direction is trivial, since there is no eliminator to express:
\[
  N \defeq N
\]

\sectionwithtoc{Substructural type system}

A~type system created with some of the~structural rules missing is called
a~\emph{substructural} type system. We can identify several type systems with
interesting properties by choosing which structural rules to omit from
the~system:
\begin{itemize}
  \item an~\emph{unrestricted} type system uses all three structural rules,
    Exchange, Contraction, and Weakening;
  \item an~\emph{affine} type system omits the~Contraction rule and uses only
    the~Exchange and Weakening rules;
  \item a~\emph{relevant} type system omits the~Weakening rule and uses only
    the~Exchange and Contraction rules;
  \item a~\emph{linear} type system omits both the~Contraction and Weakening
    rules and uses only the~Exchange rule;
  \item an~\emph{ordered} type system omits all three structural rules.
\end{itemize}
We will focus on a~linear type system, but rather then getting rid of
Contraction and Weakening outright, we want to formulate a~type system, that
still lets us embed within it the~unrestricted type system we developed in
the~\nameref{sec:stlc} section, while also giving us the~control over when
Contraction and Weakening can be used.

Recall the~problematic judgments (\ref{eq:linear_examples}) we saw at
the~beginning of this chapter and notice how the~deduction trees make use of
the~structural rules:
\begin{mathpar}
  \inferrule*[Right=Cont]
  {
    \inferrule*[Right=$\times$-I]
    {
      \inferrule*[right=Id]
      { }
      {x \is{} S \vdash x \is{} S} \\
      \inferrule*[Right=Id]
      { }
      {x \is{} S \vdash x \is{} S}
    }
    {
      x \is{} S, x \is{} S \vdash (x, x) \is{} (S \times S)
    }
  }
  {x \is{} S \vdash (x, x) \is{} (S \times S)}

  \inferrule*[Right=Weak]
  {
    \inferrule*[Right=Id]
    { }
    {z \is{} T \vdash z \is{} T}
  }
  {y \is{} S, z \is{} T \vdash z \is{} T}
\end{mathpar}
We can now see that without the~\ir{Cont} and \ir{Weak} rules, we would be
unable to deduce these judgments.

Similarly, with the~\ir{Cont} and \ir{Weak} rules missing, we cannot establish
the~equivalence between the~two distinct ways of defining the~pair rules; thus,
in linear type systems, \ir{$\times$-I} and \ir{$\times$-E} define a~different
kind of pair from \ir{$\times$-I$'$}, \ir{$\times$-E$'_1$}, and
\ir{$\times$-E$'_2$}. To distinguish between these, we will use two different
conectives, $\otimes$ for the~former pair definition and $\with$ for the~latter.
The~same is true for the~unit type, so in linear type systems we will have two
different unit types, the~original unit type denoted by \1 and the~alternative
denoted by $\top$.

There are type systems that combine linear and dependent
types~\citep{cervesato_pfenning_2002, krishnaswami_et_al_2015}; we will,
however, first describe a~linear type system with simple types and only in
\autoref{cha:qtt}, combine the~notions of linear and dependent types.

\sectionwithtoc{Linear types}

Linear types are an~extension of the~simple types in \autoref{def:simple_type}.

\begin{definition}
  Assume that we are given a~finite or infinite sequence of symbols called
  \emph{atomic types}; then we define \emph{linear types} inductively: every
  atomic type is a~linear type and if $S$ and $T$ are linear types, then
  the~following are also linear types:
  \begin{enumerate}
    \item a~\emph{linear function type} $S \multimap T$;
    \item a~\emph{multiplicative pair type} $S \otimes T$;
    \item an~\emph{additive pair type} $S \with T$;
    \item an~\emph{exponential type} $\oc S$.
  \end{enumerate}
\end{definition}

Without Contraction and Weakening the~type connectives have very different
meaning from their simple-type analogues. A~linear function $S \multimap T$ can
be read as ``consume $S$ yielding $T$.'' As noted previously, there are now two
distinct ways of defining a~pair type; these are written $S \otimes T$,
pronounced ``both $S$ and $T$,'' and $S \with T$, pronounced ``choose from $S$
and $T$.'' The~new type $\oc S$, pronounced ``of course $S$,'' is used to
indicate whether Contraction or Weakening may be used. As we mentioned, our
atomic types contain at least the~two unit types, \1 and $\top$.

\sectionwithtoc{Linear terms \& contexts}

Our particular linear type system will be based on Logic of
Unity~\citep{girard_1993}, which is a~refinement of linear logic. It simplifies
some things by using two forms of assumptions; the~first form, called
\emph{intuitionistic}, allows Contraction and Weakening, but the~second form,
called \emph{linear}, does not. We will split the~context of our typing
judgments into two parts, separated by a~vertical bar, with intuitionistic
assumptions in the~first part and linear in the~second:
\[
  x_1 \is{} S_1, \dotsc, x_n \is{} S_n \mid y_1 \is{} T_1, \dotsc, y_m \is{} T_m
    \vdash M \is{} U.
\]

Like before, $\diamond$ denotes an~empty list of assumptions and $\Gamma,
\Delta$ range over lists of zero or more assumptions.

There is a~way to define a~sound system of linear logic that does not use
intuitionistic assumptions, such as the~one described by
\citet{benton_et_al_1993}, but the~resulting system has considerably more
complex deduction rules.

Like with the~\nameref{sec:stlc}, the~\lc with linear types has an~inductive
definition of terms and a~set of judgment rules that assign types to these
terms.

\begin{definition}
  Assume that we are given an~infinite sequence of expressions $\mathbf{v}_0,
  \mathbf{v}_1, \mathbf{v}_2, \dots$ called \emph{variables}. The~set of
  \emph{\lts} is defined inductively; if $x$ and $y$ are variables, $S$ and $T$
  are types, and $M$ and $N$ are \lts, then:
  \begin{enumerate}
    \item every variable is a~\lt;
    \item \emph{linear function} abstraction and application are \lts:
      \begin{mathpar}
        \lam {x \is{} S} M \and M \: N
      \end{mathpar}
    \item \emph{multiplicative pair} and its eliminator are \lts:
      \begin{mathpar}
        \mpair M N \and \letpair{S \times T} x y M N
      \end{mathpar}
    \item \emph{multiplicative unit} and its eliminator are \lts:
      \begin{mathpar}
        \munit \and \letu M N
      \end{mathpar}
    \item \emph{additive pair} and its eliminators are \lts:
      \begin{mathpar}
        \apair M N \and \fst M \and \snd M
      \end{mathpar}
    \item \emph{additive unit} is \lt:
      \begin{mathpar}
        \aunit
      \end{mathpar}
    \item \emph{exponential} and its eliminator are \lts:
      \begin{mathpar}
        \oc M \and \letoc x M N
      \end{mathpar}
  \end{enumerate}
\end{definition}
The~syntax for the~linear functions and for the~multiplicative pair and unit
is identical to the~\nameref{sec:stlc}, while the~additive pair and unit
corresponds to the~alternative formulations from the~beginning of this chapter.
These no longer coincide due to the~missing structural rules.

The~exponential terms are terms for which the~structural rules are still
available, thus we can embed the~intuitionistic type system into the~linear type
system.

\sectionwithtoc{Linear-type \lc}\label{sec:ltlc}

Except for the~separation of intuitionistic and linear assumptions into two
parts, the~typing rules for $\multimap, \otimes$, and $\with$ are identical to
the~typing rules for $\to, \times$, and the~alternative rules for $\times$,
respectively.

The~axioms and the~structural rules are the~same; they just work on the~two-part
assumptions list:
\begin{mathpar}
  \inferrule*[right=Id$_1$]
  { }
  {x \is{} S \mid \diamond \vdash x \is{} S}

  \inferrule*[right=Id$_2$]
  { }
  {\Gamma \mid x \is{} S \vdash x \is{} S}

  \inferrule*[right=Ex]
  {\Gamma_1, \Gamma_2 \mid \Delta_1, \Delta_2 \vdash M \is{} S}
  {\Gamma_2, \Gamma_1 \mid \Delta_2, \Delta_1 \vdash M \is{} S}

  \inferrule*[right=Cont]
  {\Gamma, y \is{} S, z \is{} S \mid \Delta \vdash M \is{} T}
  {\Gamma, x \is{} S \mid \Delta \vdash M[x/y][x/z] \is{} T}

  \inferrule*[right=Weak]
  {\Gamma \mid \Delta \vdash M \is{} T}
  {\Gamma, x \is{} S \mid \Delta \vdash M \is{} T}
\end{mathpar}
With Contraction and Weakening available in the~intuitionistic part of
the~context, the~intuitionistic assumption $x \is{} S$ can be thought of as
supplying an~unlimited number of occurrences of $S$; with Contraction used, we
can supply multiple occurrences of $S$, and with Weakening, we can supply no
occurrences. The~linear assumption $x \is{} S$, on the~other hand, supplies
exactly one occurrence of $S$.

In the~introduction rule for linear functions, the~binding variable referencing
the~function parameter forms a~linear hypothesis in the~premise; hence,
the~function argument must be used in $M$. The~elimination rule states that
the~function application must consume the~same resources as both the~function
and its argument put together:
\begin{mathpar}
  \inferrule*[right=$\multimap$-I]
  {\Gamma \mid \Delta, x \is{} S \vdash M \is{} T}
  {\Gamma \mid \Delta \vdash (\lam {x \is{} S} M) \is{} S \multimap T}

  \inferrule*[right=$\multimap$-E]
  {
    \Gamma \mid \Delta_1 \vdash M \is{} S \multimap T \\\\
    \Gamma \mid \Delta_2 \vdash N \is{} S
  }
  {\Gamma \mid \Delta_1, \Delta_2 \vdash M \: N \is{} T}
\end{mathpar}

In the~rules for terms of multiplicative pair types, the~introduction rule is
similar to function application; the~resources consumed by the~derived term
equal the~resources consumed by the~term's constituents. The~elimination rule
deconstructs the~pair via the~\emph{let} term and binds both components to new
variables. The~linearity of the~two new hypotheses means that the~variables must
both be used in $N$:
\begin{mathpar}
  \inferrule*[right=$\otimes$-I]
  {
    \Gamma \mid \Delta_1 \vdash M \is{} S \\\\
    \Gamma \mid \Delta_2 \vdash N \is{} T
  }
  {
    \Gamma \mid \Delta_1, \Delta_2 \vdash \mpair M N \is{} S \otimes T
  }

  \inferrule*[right=$\otimes$-E]
  {
    \Gamma \mid \Delta_1 \vdash M \is{} S \otimes T \\\\
    \Gamma \mid \Delta_2, x \is{} S, y \is{} T \vdash N \is{} U
  }
  {
    \Gamma \mid \Delta_1, \Delta_2
      \vdash \letpair {S \times T} x y M N \is{} U
  }
\end{mathpar}

In the~additive pairs, both elements have to use all the~linear resources and
the~elimination rules ensure that only one of the~elements can be extracted:
\begin{mathpar}
  \inferrule*[right=$\with$-I]
  {
    \Gamma \mid \Delta \vdash M \is{} S \\
    \Gamma \mid \Delta \vdash N \is{} T
  }
  {\Gamma \mid \Delta \vdash \apair M N \is{} S \with T} \\

  \inferrule*[right=$\with$-E$_1$]
  {\Gamma \mid \Delta \vdash M \is{} S \with T}
  {\Gamma \mid \Delta \vdash \fst M \is{} S}

  \inferrule*[right=$\with$-E$_2$]
  {\Gamma \mid \Delta \vdash M \is{} S \with T}
  {\Gamma \mid \Delta \vdash \snd M \is{} T}
\end{mathpar}

Before we define the~remaining typing rules of the~system, the~logical rules for
the~unit types and for the~exponential type, we will examine the~differences
between the~additive and multiplicative pair types, quoting an~example due to
\citet{wadler_1993}. Under the~Curry-Howard correspondence, take $S$ to be
the~proposition ``I have 10 euro,'' $T$ to be the~proposition ``I have
a~pizza,'' and $U$ to be the~proposition ``I have a~cake.'' The~judgments
\begin{mathpar}
  \diamond \mid x \is{} S \vdash M \is{} T \and
  \diamond \mid x \is{} S \vdash N \is{} U
\end{mathpar}
express that for 10 euro we may buy a~pizza, and for 10 euro we may buy a~cake.
With the~rule \ir{$\otimes$-I}, we can derive
\begin{mathpar}
  \inferrule*[right=$\otimes$-I]
  {
    \inferrule*
    {}
    {\diamond \mid x_1 \is{} S \vdash M \is{} T} \\
    \inferrule*
    {}
    {\diamond \mid x_2 \is{} S \vdash N \is{} U}
  }
  {
    \diamond \mid x_1 \is{} S, x_2 \is{} S
      \vdash \mpair M N \is{} T \otimes U
  }
\end{mathpar}
meaning that for 20 euro we can buy both a~pizza and a~cake. With the~rule
\ir{$\with$-I} we can derive
\begin{mathpar}
  \inferrule*[right=$\with$-I]
  {
    \inferrule*
    {}
    {\diamond \mid x \is{} S \vdash M \is{} T} \\
    \inferrule*
    {}
    {\diamond \mid x \is{} S \vdash N \is{} U}
  }
  {
    \diamond \mid x \is{} S \vdash \apair M N \is{} T \with U
  }
\end{mathpar}
meaning that for 10 euro we can buy whichever we choose from a~pizza or a~cake.

Note that $T \with U$ represents our \emph{choice} of either pizza or cake,
which is different from \emph{having} one of pizza or cake. To say that we have
either a~pizza or a~cake, we would use the~sum type $T \oplus U$. We have not
formally presented the~sum type $T \oplus U$ before, because we shall not use it
outside of this example and we mention it here only to point out the~difference
from $T \with U$. With the~sum type $T \oplus U$ we can derive:
\begin{mathpar}
  \inferrule*[right=$\oplus$-I$_1$]
  {\diamond \mid x \is{} S \vdash M \is{} T}
  {\diamond \mid x \is{} S \vdash M \is{} T \oplus U}

  \inferrule*[right=$\oplus$-I$_2$]
  {\diamond \mid x \is{} S \vdash N \is{} U}
  {\diamond \mid x \is{} S \vdash N \is{} T \oplus U}
\end{mathpar}
The~first derivation means that if 10 euro buys us a~pizza, then it also buys us
a~pizza or a~cake. The~second derivation is analogous, starting with a~cake.

Now consider the~implications of having a~pair as a~hypothesis. Take $V$ to be
the~proposition ``I am happy.'' Then the~judgments
\begin{mathpar}
  \diamond \mid x \is{} T \otimes U \vdash M \is{} V \and
  \diamond \mid x \is{} T \with U \vdash M \is{} V \and
  \diamond \mid x \is{} T \oplus U \vdash M \is{} V
\end{mathpar}
mean that, first, I will be happy given \emph{both} a~pizza and a~cake; second,
I will be happy given \emph{my choice} from a~pizza and a~cake; and, third, I
will be happy given \emph{either} a~pizza or a~cake, I do not care which.

With this intuition, we can examine how the~pair types relate to each other.
Neither of the~following judgments is provable:
\begin{mathpar}
  \diamond \mid x \is{} T \otimes U \vdash M \is{} T \with U \and
  \diamond \mid x \is{} T \with U \vdash M \is{} T \otimes U
\end{mathpar}
Having a~pizza and a~cake, we cannot choose to have one of pizza or cake,
because that would mean we threw away the~other thing, which is not admissible
without the~\ir{Weak} rule~\todo{[CORRECT?]}. On the~other hand, having
the~choice of either a~pizza or a~cake does not mean we have both.

There are two proofs of $\diamond \mid x \is{} T \with U \vdash M \is{}
T \oplus U$:
\begin{mathpar}
  \inferrule*
  {
    \inferrule*[Right=$\with$-E$_1$]
    {
      \inferrule*[Right=Id$_2$]
      { }
      {\diamond \mid x \is{} T \with U \vdash x \is{} T \with U}
    }
    {\diamond \mid x \is{} T \with U \vdash \fst x \is{} T}
  }
  {\diamond \mid x \is{} T \with U \vdash \fst x \is{} T \oplus U}

  \inferrule*
  {
    \inferrule*[Right=$\with$-E$_2$]
    {
      \inferrule*[Right=Id$_2$]
      { }
      {\diamond \mid x \is{} T \with U \vdash x \is{} T \with U}
    }
    {\diamond \mid x \is{} T \with U \vdash \snd x \is{} U}
  }
  {\diamond \mid x \is{} T \with U \vdash \snd x \is{} T \oplus U}
\end{mathpar}
We can either choose the~pizza, or the~cake. There is, however, no proof of
the~converse, $\diamond \mid x \is{} T \oplus U \vdash M \is{} T \with U$;
having a~pizza or a~cake does not constitute having the~\emph{choice} of a~pizza
or a~cake.

Now we return to the~remaining typing rules of our system. The~introduction rule
for the~unit type \1 is the~same as in \nameref{sec:stlc}, while the~elimination
rule shares the~linear resources between $M$ and $N$:
\begin{mathpar}
  \inferrule*[right=\1-I]
  { }
  {\Gamma \mid \diamond \vdash \munit \is{} \1}

  \inferrule*[right=\1-E]
  {
    \Gamma \mid \Delta_1 \vdash M \is{} \1 \\
    \Gamma \mid \Delta_2 \vdash N \is{} S
  }
  {
    \Gamma \mid \Delta_1, \Delta_2 \vdash \letu M N \is{} S
  }
\end{mathpar}

The~$\top$ unit type only has an~introduction rule, as a~value of this type
cannot be eliminated:
\begin{mathpar}
  \inferrule*[right=$\top$-I]
  { }
  {\Gamma \mid \diamond \vdash \aunit \is{} \top}
\end{mathpar}

Lastly, the~introduction and elimination rules of the~exponential type. Notice
that, in this system, the~logical rules have so far all been defined in terms of
linear assumptions. To make use of our intuitionistic assumptions, we need to
turn them into linear ones, which is done using the~\oc{} connective; in fact,
an~intuitionistic assumption $x \is{} S$ is in a~sense equivalent to linear
$\oc x \is{} \oc S$.

The~introduction rule is applicable when the~context contains only
intuitionistic assumptions and it states that if we can prove $M \is{} S$ in
this context, we can also prove $\oc M \is{} \oc S$ in the~same context;
meaning, if we can derive a~single instance of $S$ from the~unlimited resources
$\Gamma$, we can also derive an~unlimited number of instances of $S$.

The~elimination rule states that we can use an~instance of type $\oc S$ to
satisfy the~intuitionistic assumption $x \is{} S$:
\begin{mathpar}
  \inferrule*[right=$\oc{}$-I]
  {\Gamma \mid \diamond \vdash M \is{} S}
  {\Gamma \mid \diamond \vdash \oc M \is{} \oc S}

  \inferrule*[right=$\oc{}$-E]
  {
    \Gamma \mid \Delta_1 \vdash M \is{} \oc S \\
    \Gamma, x \is{} S \mid \Delta_2 \vdash N \is{} T
  }
  {
    \Gamma \mid \Delta_1, \Delta_2 \vdash \letoc x M N \is{} T
  }
\end{mathpar}

The~difference between intuitionistic assumption $x \is{} S$ and linear
assumption $\oc x \is{} \oc S$ is that the~linear assumption still must be
used exactly once in the~derivation tree, while the~intuitionistic one may be
used any number of times. They are equivalent, however, in the~sense that
$\Gamma \mid \Delta, x \is{} \oc S \vdash M \is{} T$ is provable if
and only if $\Gamma, x \is{} S \mid \Delta \vdash N \is{} T$ is
provable. This is shown by the~following derivation trees:
\begin{mathpar}
  \inferrule*[Right=$\multimap$-E]
  {
    \inferrule*[right=Weak]
    {
      \inferrule*[Right=$\multimap$-I]
      {\Gamma \mid \Delta, x \is{} \oc S \vdash M \is{} T}
      {
        \Gamma \mid \Delta
          \vdash (\lam {x \is{} \oc S} M) \is{} \oc S \multimap T
      }
    }
    {
      \Gamma, x \is{} S \mid \Delta
        \vdash (\lam {x \is{} \oc S} M) \is{} \oc S \multimap T
    } \\
    \inferrule*[Right=$\oc{}$-I]
    {
      \inferrule*[Right=Weak]
      {
        \inferrule*[Right=Weak]
        {
          \inferrule*[Right=Id$_1$]
          { }
          {x \is{} S \mid \diamond \vdash x \is{} S}
        }
        {\vdots}
      }
      {\Gamma, x \is{} S \mid \diamond \vdash x \is{} S}
    }
    {\Gamma, x \is{} S \mid \diamond \vdash \oc x \is{} \oc S}
  }
  {
    \Gamma, x \is{} S \mid \Delta
      \vdash (\lam {x \is{} \oc S} M) \: \oc x \is{} T
  }

  \inferrule*[Right=$\oc{}$-E]
  {
    \inferrule*[right=Id$_2$]
    { }
    {\diamond \mid x \is{} \oc S \vdash x \is{} \oc S} \\
    \Gamma, x \is{} S \mid \Delta \vdash N \is{} T
  }
  {
    \Gamma \mid \Delta, x \is{} \oc S \vdash \letoc x x N \is{} T
  }
\end{mathpar}

The~reason why we indeed need separate intuitionistic assumptions $x \is{} S$
and cannot use $\oc x \is{} \oc S$ instead is explained by \citet{wadler_1993}.

