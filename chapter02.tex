\chapter{Type Systems}\label{cha:typesystems}

In mathematics, the~definition of a~particular function usually specifies
the~kind of inputs it accepts, and the~kind of output it produces. For example,
at the~start of the~previous chapter, we saw the~function $f(x) = x + y$ applied
to an~integer 0. Although we did not mention the~domain nor the~range of
the~function then, we can now pick some and say that $f$ is a~function from
integers to integers, or
\[
  f: \Int \to \Int.
\]
Corresponding to the~mathematical intuition, the~pure \lc can be modified by
attaching expressions called \emph{types} to terms, like labels to denote their
intended input and output. While conventionally the~term \emph{type} is used to
refer to an~abstraction of a~set of values, generally, a~type can be any
property of a~term, that we can establish without executing it.

We are looking to formulate the~procedure of assigning the~type annotations to
$\lambda$-terms, such that the~terms for which this procedure fails will be
deemed invalid. Hence, thanks to the~type system we are able to reject faulty
programs that exhibit unwanted behaviour before they are executed.
As a~consequence of the~\emph{Halting Problem} though, the~conclusions a~type
system can make are necessarily just approximations. Either it must reject
programs that might have run without an~error, or accept programs that may error
when executed. With more expressive type systems, we are able to more precisely
decide whether to reject a~given program.

In this chapter, we will formulate gradually more expressive type systems,
but to do that effectively, we need to introduce a~bit of new notation first.

\sectionwithtoc{Natural deduction}\label{sec:natural_deduction}

Conventionally, type systems are presented using \emph{natural deduction}, which
was first introduced by Gentzen as a~new formulation of
logic~\cite{gentzen_1935}. Natural deduction is a deductive system and is thus
specified by a~collection of \emph{judgments} and a~collection of \emph{rules}.
A~rule takes a~list of judgments and produces a~new judgment.

\emph{Judgment} is a~proposition in the~metalanguage. For example, in
first-order logic, judgment can say that a~certain string of symbols $A$ is
a~\emph{well-formed proposition}, written \emph{$A$ prop}. Another judgment may
say that these symbols form a~\emph{proposition that is true}, written
\emph{$A$ true}.

A~\emph{rule} is written as a~horizontal line with its \emph{premises}, or
\emph{antecedents}, above the~line and its \emph{conclusion}, or
\emph{consequent}, below the~line. There can be zero or more premises, but
conclusion is exactly one. The~interpretation of a~rule is that if all premises
hold, then the~conclusion follows. An~example of a~rule may be the~rule of
implication elimination, or \emph{modus ponens}, which can be stated as:
\begin{mathpar}
  \inferrule*[right=Mp]
  {P \supset Q \\ P}
  {Q}.
\end{mathpar}
It says that if ``$P$ implies $Q$,'' written $P \supset Q$, and ``$P$,'' then
``$Q$.'' Note that some rules have a~name, which is attached to the~right side
of the~horizontal line. In our example the~name is \ir{Mp}.

Given a~set of rules, we are able to combine them to form a~tree, where every
node is an~instance of a~rule. We say we can \emph{deduce}, or \emph{derive},
a~formula if there exists a~tree where the~formula is at the~root and all leaves
are \emph{axioms}, that is rules that have no premises.

Natural deduction also often involves \emph{hypothetical judgments}, or
\emph{reasoning from assumptions}. To express hypothetical reasoning within
the~rules of a~deductive system, we indicate in each judgment on which
hypotheses it depends. To denote the~hypothetical judgment of $Q$ assuming $P$,
we write
\[
  P \vdash Q.
\]
The~list of assumptions on which a~judgment depends, denoted by $\Gamma$, can be
extended by appending an~assumption $P$ to the~end of the~list with a~comma:
$\Gamma, P$. We can also combine lists of assumptions; the~list that results
from appending the~assumptions from $\Gamma_2$ to the~list $\Gamma_1$ is written
$\Gamma_1, \Gamma_2.$

For example, the~derivation tree of the~judgment that given that $P$ implies $Q$
and given $P$, one may deduce both $P$ and $Q$:
\begin{mathpar}
  \inferrule*
  {
    \inferrule*
    { }
    { P \supset Q, P \vdash P } \\
    \inferrule*[right=Mp]
    {
      \inferrule*
      { }
      {P \supset Q, P \vdash P \supset Q} \\
      \inferrule*
      { }
      {P \supset Q, P \vdash P }
    }
    { P \supset Q, P \vdash Q }
  }
  { P \supset Q, P \vdash P \wedge Q }.
\end{mathpar}

Notice that in the~definition of \ir{Mp} we have elided the~ambient
assumptions that could have been added to every judgment, i.e. we could have
written:
\begin{mathpar}
  \inferrule*[right=Mp$'$]
  {\Gamma_1 \vdash P \supset Q \\ \Gamma_2 \vdash P}
  {\Gamma_1, \Gamma_2 \vdash Q}.
\end{mathpar}

\sectionwithtoc{Simple types}

We will present a~few different type systems throughout the~text, each
an~extension of some previous ones. So to start at the~beginning, we will
define a~basic type system that utilises the~\emph{simple types}.

\begin{definition}
  Assume that we are given a finite or infinite sequence of symbols called
  \emph{atomic types}; then we define \emph{simple types} as follows:
  \begin{enumerate}
    \item every atomic type is a~simple type;
    \item if $\sigma$ and $\tau$ are simple types, then $(\sigma \to \tau)$ is
      a~simple type, called a~\emph{function type};
    \item if $\sigma$ and $\tau$ are simple types, then $(\sigma \times \tau)$
      is a~simple type, called a~\emph{pair type};
  \end{enumerate}
\end{definition}

Atomic types are intended to denote a~particular set. For example we may have
the~atomic type \Int for the~set of integers, and \Bool for the~set of Boolean
values $\{ \mathtt{true}, \mathtt{false} \}$.

A~function type $(\sigma \to \tau)$ is intended to denote some given set of
\emph{functions from $\sigma$ to $\tau$}. That is, functions that accept as
input a~member of its \emph{domain}, the~set denoted by $\sigma$, and produce
an~output in its \emph{range}, a~subset of the~set denoted by $\tau$. The~exact
set of functions denoted by $(\sigma \to \tau)$ depends on the~intended use
of the~type system we are trying to build. It might be the~set of \emph{all}
functions from $\sigma$ to $\tau$, or just some subset. Some examples of simple
types are:
\begin{align*}
  &(\Int \to \Int) && \text{functions from integers to integers;} \\
  &(\Int \to \Bool) && \text{functions from integers to Boolean values;} \\
  &(\Int \to (\Int \to \Int)) && \text{functions from integers to functions;} \\
  &((\Int \to \Int) \to \Int) && \text{functions from functions to integers;} \\
  &((\Int \to \Int) \to (\Int \to \Int)) && \text{functions from functions to
    functions.}
\end{align*}
In contrast with the~$\lambda$-term application, the~function type is
right-associative and we will abbreviate accordingly:
\begin{align*}
  &\sigma \to \tau  &  &\text{for} &  &(\sigma \to \tau); \\
  &\rho \to \sigma \to \tau  &  &\text{for} &  &(\rho \to (\sigma \to \tau)); \\
  &\sigma_1 \to \dotsb \to \sigma_n \to \tau  &  &\text{for} &  
    &(\sigma_2 \to ( \dotsb \to (\sigma_n \to \tau) \dotsb )); \\
  &\sigma \times \tau  &  &\text{for} &  &(\sigma \times \tau).
\end{align*}
We do not define any order of precedence between the~function and pair types.
Note that an~expression like $\sigma \to \tau$, which contains Greek-letter type
variables, is not itself a~type. It is only a~name in the~metalanguage for
a~type of unspecified functions.

A~pair type, or a~\emph{product type}, $(\sigma \times \tau)$ denotes
the~Cartesian product of the~sets denoted by $\sigma$ and $\tau$. For example,
$((\Int \to \Int) \times \Bool)$ denotes the~set of ordered pairs $(f, b)$,
where $f$ is a~function from integers to integers, and $b$ is a Boolean value.

There is also a~dual notion of a~pair type, the~\emph{sum type} $(\sigma +
\tau)$, which is used to denote the~set of members of either $\sigma$, or
$\tau$. In this text however, we are focused on the~pair type, so we will not be
working with the~sum type further.

\sectionwithtoc{Type judgment}

When we apply the~natural deduction in the~context of types, we are able to
formulate deduction trees that assign types to terms.

\begin{definition}
  For any $\lambda$-term $M$ and type $\tau$, define a~\emph{type assignment}
  formula as
  \[
    M \is{} \tau.
  \]
\end{definition}

The~formula $M \is{} \tau$ can be read informally as `$M$ has type $\tau$', or
`we assign to $M$ the~type $\tau$'. Then, \emph{type judgment} is the~formula
\[
  \Gamma \vdash M \is{} \tau,
\]
which is defined to mean that there is a~deduction-tree, whose root formula is
$M \is{} \tau$, and whose leaves are members of $\Gamma$. When type judgment
has no assumptions, we say so explicitly by writing
\[
  \vdash M \is{} \tau.
\]

\sectionwithtoc{Simply typed \lc}\label{sec:stlc}

Generally, there are two ways we can attach types to terms. The~\emph{implicit},
or \emph{Curry-style}, semantics use the~$\lambda$-terms as we have seen them
in \autoref{def:lambda_calculus} and assign type to a~term after it had been
built. The~type then sits next to the~term. In the~other approach, called
\emph{explicit}, or \emph{Church-style}, the~term's type is a~built-in part of
the~term itself.

We expect to present complex enough systems later, for which finding the types
in the~implicit version may not be computable~\cite{wells_1999}, hence we will
use the~explicit semantics.

\begin{definition}[Typed variables]
  Assume that we are given an~infinite sequence of expressions $\mathbf{v}_0,
  \mathbf{v}_1, \mathbf{v}_2, \cdots$ called \emph{untyped variables}. We make
  \emph{typed variables $x^\tau$} by attaching type-superscripts to untyped
  variables, in such a~way that
  \begin{enumerate}
    \item \label{def:typed_var:a_item} no untyped variable receives more than
      one type;
    \item \label{def:typed_var:b_item} every type $\tau$ is attached to
      an~infinite sequence of variables.
  \end{enumerate}
\end{definition}
We call $x^\tau$ a~\emph{variable of type $\tau$}. It denotes an~arbitrary
member of the~set denoted by $\tau$.

Condition \ref{def:typed_var:a_item} ensures that there are no $x^\tau$,
$x^\sigma$ with $\tau \not\equiv \sigma$. For example, if we make a~typed
variable $x^\Int$, to denote an~arbitrary integer, we cannot also make
$x^{\Int \to \Int}$, which would denote a~function.

Condition \ref{def:typed_var:b_item} ensures that, for every type $\tau$, there
is always enough variables to denote all members of the~set that we wish to
describe.

\begin{definition}[Simply typed $\lambda$-terms]\label{def:stlc}
  The~set of \emph{typed $\lambda$-terms} is defined as follows:
  \begin{enumerate}
    \item all typed variables $x^\tau$ are typed $\lambda$-terms of type $\tau$;
    \item \label{def:stlc:b_item} if $M^{\sigma \to \tau}$ and $N^\sigma$ are
      typed $\lambda$-terms of types $\sigma \to \tau$ and $\sigma$
      respectively, then the~following is a~typed $\lambda$-term of type $\tau$:
      \[
        (M^{\sigma \to \tau} N^\sigma)^\tau;
      \]
    \item if $x^\sigma$ is a~variable of type $\sigma$ and $M^\tau$ is a~typed
      $\lambda$-term of type $\tau$, then the~following is a~typed
      $\lambda$-term of type $\sigma \to \tau$:
      \[
        (\lam {x^\sigma} {M^\tau})^{\sigma \to \tau};
      \]
    \item \label{def:stlc:d_item} if $M^\sigma$ and $N^\tau$ are typed
      $\lambda$-terms of types $\sigma$ and $\tau$ respectively, then
      the~following is a~typed $\lambda$-term of type $\sigma \times \tau$:
      \[
        \mpair {M^\sigma} {N^\tau} ^{\sigma \times \tau};
      \]
    \setcounter{stlc_counter}{\value{enumi}}
    \item \label{def:stlc:e_item} if $x^\sigma$ and $y^\tau$ are variables of
      types $\sigma$ and $\tau$ respectively, and $M^{\sigma \times \tau}$ and
      $N^\rho$ are typed $\lambda$-terms of types $\sigma \times \tau$ and
      $\rho$ respectively, then the~following is a~typed $\lambda$-term of type
      $\rho$:
      \[
        (\textrm{let} \: \mpair {x^\sigma} {y^\tau} = M^{\sigma \times \tau} \:
        \textrm{in} \: N^{\rho})^\rho.
      \]
\end{enumerate}
\end{definition}

The~typed term $M^\tau$ is intended to denote a~member of the~set denoted by
$\tau$. For example, in \ref{def:stlc:b_item}, if $M^{\sigma \to \tau}$ denotes
a~function $\phi$ from $\sigma$ to $\tau$, and $N^\tau$ denotes a~member $a$ of
$\sigma$, then the~term $(M^{\sigma \to \tau} N^\sigma)^\tau$ denotes $\phi(a)$,
which is of type $\tau$.

Note that in \ref{def:stlc:e_item}, the~pair $(x,y)$ is just a~part of
the~syntax; it is not an~instance of a~term constructing a~pair like in
\ref{def:stlc:d_item}.

When writing typed terms, some of the~type-superscripts can be omitted if it is
obvious from the~context what they should be. Other notation conventions will be
the~same as in \autoref{cha:lambdacalculus}.

For example, for every pair of types $\sigma$ and $\tau$,
\[
  (\lam {x^\sigma} {(\lam {y^\tau} {\mpair {x^\sigma} {y^\tau} ^{\sigma \times
    \tau}})^{\tau \to (\sigma \times \tau)}})^{\sigma \to (\tau \to (\sigma
    \times \tau))}
\]
is a~typed term and it can be written as
\begin{align*}
  &(\lam {xy} {\mpair x y})^{\sigma \to (\tau \to (\sigma \times \tau))},  &
  &\text{or}  &  &\lam {x^\sigma y^\tau} {\mpair x y}.
\end{align*}

The~substitution $[N^\sigma/x^\sigma](M^\tau)$ is defined in the~same way as
the~substitution of pure $\lambda$-terms in \autoref{def:substitution}.
Note that we do not define substitution $[N^\rho/x^\sigma](M^\tau)$ when 
$\rho \not\equiv \sigma$. \todo{[NEEDS REWRITE TO MENTION PAIRS]}

Finally, we will use the~simply typed \lc to construct a~formal theory where
the~deduction assigns types to terms. This is called a~\emph{type-assignment
system}. A~system with simply typed $\lambda$-terms usually has the~following
rules:
\begin{mathpar}
  \inferrule*[right=Var]
  { }
  {\Gamma_1, x \is{} \tau, \Gamma_2 \vdash x \is{} \tau} \\

  \inferrule*[right=App]
  {\Gamma \vdash M \is{} \sigma \to \tau \\ \Gamma \vdash N \is{} \sigma}
  {\Gamma \vdash M \: N \is{} \tau}

  \inferrule*[right=Lam]
  {\Gamma, x \is{} \sigma \vdash M \is{} \tau}
  {\Gamma \vdash \lam {x^\sigma} M \is{} \sigma \to \tau}

  \inferrule*[right=Let]
  {
    \Gamma \vdash M \is{} \sigma \times \tau \\
    \Gamma, x \is{} \sigma, y \is{} \tau \vdash N \is{} \sigma
  }
  {
    \Gamma \vdash \textrm{let} \: \mpair {x^\sigma} {y^\tau} = M \: \textrm{in}
    \: N \is{} \rho
  }

  \inferrule*[right=Pair]
  {\Gamma \vdash M \is{} \sigma \\ \Gamma \vdash N \is{} \tau}
  {\Gamma \vdash \mpair M N \is{} \sigma \times \tau}.
\end{mathpar}

\sectionwithtoc{Dependent types}

The Church-style system from the~previous chapter has lost some of
the~versatility of the~pure \lc. The~term $\lam {x^\tau} {x^\tau}$ represents
the~one operation of doing nothing, the~identity function, yet for different
types $\tau$ this one operation is represented by distinct terms. We want our
formalism to keep track of the~fact that identity does the~same thing regardless
of the~type. An~ability to universally quantify the~types, which we will gain
with the~introduction of dependent types, will allow us to express this; then
the~type of an~identity function will, in effect, be `for all types $\tau$,
$\tau \to \tau$'.

To properly talk about dependent types, we feel we need to first touch on one
important observation; it is the~correspondence between computer programs and
mathematical proofs, known as the~\emph{Curry–Howard isomorphism}. In 1934,
Curry observed that every type of a~function, $\sigma \to \tau$, can be read as
a~proposition of an~implication, $\sigma \supset \tau$~\cite{curry_1934}.
Then there exists a~function of a~given type if and only if, the~corresponding
proposition is provable. Curry and Feys later extended the~observation to also
include the~correspondence between terms and proofs~\cite{curry_1958}. Finally,
in 1969, in manuscript not published until 1980~\cite{howard_1980}, Howard
described the relation between the~natural deduction and the~simply typed \lc,
as well as shown that the~simplification of proofs corresponds to the~evaluation
of programs. Wadler provides a~detailed yet approachable introduction to
the~topic~\cite{wadler_2015}. Now we will generalise the~simple types into their
dependent counterparts and use the~Curry–Howard correspondence to form a~better
intuition around them.

The~dependent analogue of the~type $\sigma \to \tau$ is the~\emph{dependent
function type}, in the~academic literature usually written as ${(\Pi x \is{}
\sigma \, . \: \tau(x))}$, but we will write:
\[
  \dep x \sigma {\tau(x)}.
\]
Here, $\sigma$ is a~type, but $\tau(x)$ is a~function whose argument is of type
$\sigma$ and it produces value that is a~type. Thus, a~term of type $\dep x
\sigma {\tau(x)}$ represents function that for an~argument $N$ of type $\sigma$
produces a~value of type $\tau(N)$. In the~special case that $\tau(x)$ is
a~constant function whose value is always the~type $\rho$, $\dep x \sigma
{\tau(x)}$ is the~type $\sigma \to \rho$.

On the~logic side, the~dependent function type corresponds to a~universal
quantification; hence, we can read it as ``for all $x$ in $\sigma$, $\tau(x)$''.
The~$\Pi$ in the~notation of the~dependent function type references the~fact
that we can interpret it as a~\emph{product} of $\tau(x)$, for every $x \in
\sigma$; that is, a~value of type $\dep x \sigma {\tau(x)}$ can be interpreted
as a~tuple of size $|\sigma|$, where the~$x$th element is of type $\tau(x)$.

The~dependent analogue of type $\sigma \times \tau$ is the~\emph{dependent pair
type}, again, usually written as ${(\Sigma x \is{} \sigma \, . \: \tau(x))}$,
but we will write:
\[
  (x \is{} \sigma) \times \tau(x).
\]
The value of this type is a~pair $\mpair x y$, where $x$ is of type $\sigma$ and
$y$ is of type $\tau(x)$; hence, the~type of the~second element is determined by
the~value of the~first.

This type corresponds to the~existential quantification in logic. It uses
$\Sigma$ in its notation because we can interpret it as a~\emph{disjoint union}
of all $\tau(x)$, with an~index from $\sigma$; meaning, a~value of type $(x
\is{} \sigma) \times \tau(x)$ can be interpreted as a~member of the~set
\[
  \bigcup_{x \in \sigma} \{ \mpair x y \mid y \in \tau(x) \}.
\]
By having a~value of this dependent pair type, we are therefore \emph{proving}
that ``there exists an~$x \in \sigma$, such that $\tau(x)$ is inhabited,'' where
\emph{inhabited} means that the~type contains some value.

In dependent types, terms can occur in place of types so we must scrap
the~distinction between terms and types. Thus, we will define
\emph{pseudo-terms} which will subsume both the~terms from \autoref{def:stlc}
and types.

Now that types will be just a~special kind of terms, we will make two changes to
the~notation of the~typed $\lambda$-abstraction. First, we will write
\[
  \lam {x \is{} \tau} M,
\]
instead of $\lam {x^\tau} M$, analogously to the~way we write the~dependent
function type. Second, to denote types, we will use $S, T$, etc. instead of
$\sigma, \tau.$

\sectionwithtoc{Pseudo-terms}

For the~following type systems, we will use the~general notion of
\emph{pseudo-terms}. First, we define a~generous \emph{pseudo-terms} grammar and
then we will construct typing rules, that identify the~well-typed terms within
pseudo-terms.

\begin{definition}\label{def:pseudo-term}
  The~set of all \emph{pseudo-terms} is defined inductively as follows:
  \begin{enumerate}
    \item every variable is a~pseudo-term;
    \item every atomic type is a~pseudo-term;
    \item if $M$ and $N$ are pseudo-terms, then $(M N)$ is a~pseudo-term;
    \item if $x$ is a~variable, and $M$ and $N$ are pseudo-terms, then
      $(\lam {x \is{} M} N)$ is a~pseudo-term.
    \item if $M$ and $N$ are pseudo-terms and $x$ is a~variable which does not
      occur free in $M$, then $\dep x M N$ is a~pseudo-term.
    \item if $M$ and $N$ are pseudo-terms, then $\mpair M N$ is a~pseudo-term.
    \item if $M$ and $N$ are pseudo-terms and $x$ is a~variable which does not
      occur free in $M$, then $(x \is{} M) \times N$ is a~pseudo-term.
  \end{enumerate}
\end{definition}

The~reduction on pseudo-terms is defined by replacing the~$\beta$-reduction on
pure $\lambda$-terms with one of the~form: \todo{[WHAT ABOUT PAIRS?]}
\[
  (\lam {x \is{} S} M) N \triangleright [N/x]M.
\]

Barendregt~\cite{barendregt_1993} introduced a~group of eight typing systems,
which are known as the~\emph{$\lambda$-cube} due to their mutual relationships.
All these systems are based on the~pseudo-terms, the~dependent function type,
and two atomic types, $\star$ and $\square$, related by the~rule
\begin{mathpar}
  \inferrule*
  { }
  {\vdash \star \is{} \square}.
\end{mathpar}

The~two distinct atomic types are required for the~type system to by
\emph{consistent} when interpreted as a~logic under the~Curry–Howard
correspondence~\todo{[CITE?]}. A~logic is inconsistent when we can derive
a~contraction in it, which for types means that every type is inhabited.
Because we do not utilise the~logic interpretation, we can simplify here and
keep on using a~single atomic type.

For our atomic type we choose \univ, called the~\emph{universe}, with
the~following rule:
\begin{mathpar}
  \inferrule*
  { }
  {\vdash \univ \is{} \univ}.
\end{mathpar}

\sectionwithtoc{Precontexts}

With dependent types in mind, we need to revisit the~semantics of the~judgement
$\Gamma \vdash M \is{} S$. In typing systems, the~list of assumptions $\Gamma$,
henceforth called \emph{context}, contains only type assignments. We, however,
need to make sure that these assignments are well-formed; that is, for every
$x \is{} S$ in $\Gamma$, we want to make sure that $S$ is a~proper type. For
example, we want to reject the~list:
\[
  a \is{} \univ, \; f \is{} \dep x b a,
\]
because we do not know the~type of the~variable $b$. A~valid context would be:
\[
  a \is{} \univ, \; b \is{} \univ, \; f \is{} \dep x b a.
\]
Hence, we first define \emph{precontexts} and then identify the~contexts within.

\begin{definition}
  A~\emph{precontext} is a~sequence of formulas of the~form
  \[
    x_1 \is{} S_1, \dotsc , x_n \is{} S_n,
  \]
  where $x_1, \dotsc , x_n$ are distinct variables and $S_1, \dotsc, S_n$ are
  pseudo-terms.
\end{definition}

Empty precontext is denoted by $\diamond$, or it is omitted when no confusion is
likely.

\sectionwithtoc{Dependently typed \lc}

Using pseudo-terms and precontexts, we are able to define a~type-assignment
system, that assigns the~dependent types, and is thus more expressive then
the~simply typed \lc, since we can model the~simple type $S \to T$ by defining
it as $\dep x S T$, where $x \notin \fv T$.

The~increased expressiveness of dependent types lies in the~ability to quantify
over \univ. We will see an~example of this after we formalise the~type judgment
rules.

The~rules that identify contexts within precontexts and terms within
pseudo-terms are mutually recursive. Contexts are identified by the~judgement
$\Gamma \vdash$, defined by the~following rules:
\begin{mathpar}
  \inferrule*
  { }
  {\diamond \vdash}

  \inferrule*
  {\Gamma \vdash \\ \Gamma \vdash S : \univ}
  {\Gamma, x \is{} S \vdash}
\end{mathpar}
and terms are identified by the~judgement $\Gamma \vdash M \is{} S$, defined by
rules:
\begin{mathpar}
  \inferrule*[right=Fun]
  {\Gamma \vdash S \is{} \univ \\ \Gamma, x \is{} S \vdash T \is{} \univ}
  {\Gamma \vdash \dep x S T \is{} \univ}

  \inferrule*[right=App$'$]
  {
    \Gamma \vdash M \is{} \dep x S T \\
    \Gamma \vdash N \is{} S \\
  }
  {\Gamma \vdash M \: N \is{} [N/x]T}

  \inferrule*[right=Lam$'$]
  {
    \Gamma, x \is{} S \vdash M \is{} T \\
    \Gamma \vdash \dep x S T \is{} \univ
  }
  {\Gamma \vdash (\lam {x \is{} S} M) \is{} \dep x S T}
\end{mathpar}

Now we can look at an~example of quantification over \univ. For example,
although
\begin{equation}\label{eq:pi_type}
  s \is{} \univ \vdash \dep x s s \is{} \univ
\end{equation}
and
\begin{equation}\label{eq:id_lam}
  s \is{} \univ \vdash (\lam {x \is{} s} x) \is{} \dep x s s
\end{equation}
can be expressed with just simple types, to be able to successfully derive
the~type of an~identity function that works for any type, we need dependent
function types. Meaning, we can combine judgments (\ref{eq:pi_type}) and
(\ref{eq:id_lam}) and deduce:
\begin{mathpar}
  \inferrule*[right=Lam$'$]
  {
    \text{(\ref{eq:id_lam})} \\
    \inferrule*[right=Fun]
    {
      \inferrule*
      { }
      {\vdash \univ \is{} \univ} \\
      \text{(\ref{eq:pi_type})}
    }
    {\vdash \dep s \univ {\dep x s s} \is{} \univ}
  }
  {
    \vdash (\lam {s \is{} \univ} {\lam {x \is{} s} x}) \is{}
      \dep s \univ {\dep x s s}
  }.
\end{mathpar}

Now $\lam {x \is{} s} x$ is the~identity function on type $s$. Hence
\[
  \lam {s \is{} \univ} {\lam {x \is{} s} x}
\]
is a~function which, when applied to any type of type \univ, gives the~identity
function over that type.

