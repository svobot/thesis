\chapter{Quantitative Type Theory}\label{cha:qtt}

In the~type systems in the~previous two chapters, we have seen variables used
for two distinct purposes: it is either used during type checking and not
available for the~computation of the~program, for example, the~term $\Int \times
x$ depends on $x$ to form a~valid type; or, in the~linear type system, as
a~resource that is consumed during the~program runtime.

The~linear type system in the~previous chapter tracks the~distinction between
variable purposes by separating the~context into two parts, intuitionistic and
linear,
\[
  x_1 \is{} S_1, \dotsc, x_n \is{} S_n \mid y_1 \is{} T_1, \dotsc, y_m \is{} T_m
    \vdash M \is{} U.
\]
The~unrestricted variables $x_i$ may be used zero, one, or more times.
The~linear variables $y_i$, however, must be used exactly once. Type systems
that adhere to the~linear logic interpretation, which equates resource usage
with the~presence of a~variable in (the~second part of) a~context, are
restricted in what kinds of \emph{use} they can express: either variable is used
or it is not.

Another kind of restriction is that our linear type system does not integrate
dependent types. Although we mentioned systems that combine the~two, they do so
by allowing the~linear variable types to dependent on intuitionistic variables.
This solution, however, would still not allow us to formulate types that depend
on linear variables.

Our new type system removes these two restrictions. The~context does not get
separated into two parts, instead we track how a~variable can be used by
annotating it with an~element of some semiring. Several authors have used
semiring annotations of variables to attach information on how they can be used,
but McBride~\cite{mcbride_2016} was first to use the~annotations to indicate
whether variables are available for computation or only available in
the~formation of types. Atkey fixed an~issue in McBride's idea and presented it
as the~Quantitative Type Theory (QTT)~\cite{atkey_2018}. Weirich provided
further commentary on the~topic~\cite{weirich_2020}.

\sectionwithtoc{Resource semirings}

In quantitative type theory, the~typing judgement takes the~form:
\[
  x_1 \is{\rho_1} S_1, \dotsc, x_n \is{\rho_n} S_n \vdash M \is{\sigma} T,
\]
where the~$\rho_1, \dotsc, \rho_n$ are elements of the~semiring indicating
how the~corresponding variable may be used. The~annotation $\rho_i$ tracks
the~computational usage of a~variable. The~usage $0$ indicates that the~variable
can only be used to form a~type and it has no presence in an~executing program.
The~output is annotated with usage $\sigma$, which is restricted to be either
the~$0$, or the~$1$ of the~semiring due to the~admissibility of substitution, as
Atkey shows.

\begin{definition}
  A~\emph{semiring} $\mathcal{R}$ is a~set equipped with two binary operations,
  addition ($+$) and multiplication ($\cdot$), and constants $0, 1 \in
  \mathcal{R}$, such that ($\mathcal{R}, +, 0$) is a~commutative monoid, i.e.
  for every $\rho, \pi, \phi \in \mathcal{R}$:
  \begin{align*}
    (\rho+\pi)+\phi = \rho+(\pi+\phi),&  &  &0 + \rho = \rho,&  &
    &\rho + \pi = \pi + \rho;
  \end{align*}
  ($\mathcal{R}, \cdot, 1$) is a~monoid, i.e. for every $\rho, \pi, \phi
  \in \mathcal{R}$:
  \begin{align*}
    (\rho \cdot \pi) \cdot \phi = \rho \cdot (\pi \cdot \phi),&  &
    &1 \cdot \rho = \rho \cdot 1 = \rho;
  \end{align*}
  multiplication left and right distributes over addition, i.e. for every $\rho,
  \pi, \phi \in \mathcal{R}$:
  \begin{align*}
    \rho \cdot (\pi + \phi) = (\rho \cdot \pi) + (\rho \cdot \phi),&  &
    &(\rho + \pi) \cdot \phi = (\rho \cdot \phi) + (\pi \cdot \phi);
  \end{align*}
  and for every $\rho \in \mathcal{R}$:
  \[
    0 \cdot \rho = \rho \cdot 0 = 0.
  \]
\end{definition}
As usual, we omit the~symbol $\cdot$ from notation and write $\rho\pi$ instead
of $\rho\cdot\pi$. For the~use of in annotations, we also require that
the~semiring is \emph{positive}, meaning that $\rho + \pi = 0$ implies that
$\rho = 0$ and $\pi = 0$, and has the~\emph{zero-product} property:
$\rho\pi = 0$ implies $\rho = 0$ or $\pi = 0$. These two properties are required
for the~admissibility of substitution in the~system we will define.

The~properties of semirings make $0$ well suited for identifying the~use in
types. Since adding $0$ to an~existing use, $0 + \rho = \rho$, retains
the~original, we can always combine a~computational usage with a~use in a~type;
and because $0\rho = 0$, nesting an~otherwise computational use within a~type
results in the~whole usage being noncomputational.

Semirings that form type systems with interesting properties include:
\begin{itemize}
  \item the~singleton $\{0\}$, which produces the~intuitionistic type system we
    saw in \autoref{cha:typesystems};
  \item the~semiring $\{0, 1\}$ corresponds to the~linear type system from
    \autoref{cha:linear};
  \item the~natural numbers with addition and multiplication give us
    \todo{[WHAT?]};
  \item the~zero-one-many semiring $\{0, 1, \omega\}$, where $\rho + \omega =
    \omega$ and $\omega \cdot \omega = \omega$. The~$0$ value restricts
    the~resource usage to types only, $1$ requires linear usage, and $\omega$
    has no usage restrictions.
\end{itemize}

\sectionwithtoc{Terms \& contexts}

As with previous systems, we will start with pseudo-contexts and pseudo-terms,
and then we will use the~deduction rules to identify which terms are properly
typed.

\begin{definition}
  The~set of all \emph{pseudo-terms} of a~quantitative-type \lc is defined
  inductively: every variable is a~pseudo-term; $\munit, \: \1, \: \aunit,
  \: \top$, and $\univ$ are pseudo-terms; and if $M, N$ are pseudo-terms and
  $x, y$ are variables, then the~following are pseudo-terms:
  \begin{mathpar}
    (\lam {x \is\pi M} N) \and (M N) \and \depq x \pi M N \\
    \mpair M N \and \textrm{let} \: \mpair x y = M \: \textrm{in} \: N \and
      (x \is\pi M) \otimes N \\
    \textrm{let} \: \munit \: \textrm{=} \: M \: \textrm{in} \: N \\
    \apair M N \and (\fst M) \and (\snd M) \and
      (x \is{} M) \with N
  \end{mathpar}
\end{definition}
We use the~usual linear logic notation for the~multiplicative unit type, \1.
Although there should be no confusion, note the~difference between a~semiring
element, $1$, and a~multiplicative unit type, \1.

Pseudo-contexts are now sequences of the~form:
\[
  x_1 \is{\rho_1} S_1, \dotsc, x_n \is{\rho_n} S_n,
\]
where $\rho_1, \dotsc, \rho_n$ are arbitrary elements of the~semiring. We define
scaling of pseudo-contexts by a~semiring element and an~addition of
pseudo-contexts:
\begin{definition}
  Given a~pseudo-context $\Gamma$ and a~semiring element $\pi$,
  the~\emph{scaling} $\pi\Gamma$ is defined by:
  \begin{align*}
    \pi (\diamond) = \diamond&  &  &\pi(\Gamma, x \is\rho S) = \pi\Gamma,
    x \is{\pi\rho} S
  \end{align*}
  The~\emph{addition} $\Gamma_1 + \Gamma_2$ is the~pseudo-context that contains
  $x \is\pi S$, if:
  \begin{enumerate}
    \item $x \is{\rho_1} S$ is in $\Gamma_1$ and $x \is{\rho_2} S$ is in
      $\Gamma_2$, and $\pi = \rho_1 + \rho_2$;
    \item $x \is\pi S$ is in $\Gamma_1$, and $x \is\rho T$ is not in $\Gamma_2$;
    \item $x \is\pi S$ is in $\Gamma_2$, and $x \is\rho T$ is not in $\Gamma_1$.
  \end{enumerate}
\end{definition}
The~zeroing $0\Gamma$ sets all the~annotations to $0$, due to the~semiring laws.
We do not define the~addition $\Gamma_1 + \Gamma_2$ if, for some $x \is\rho S$
in $\Gamma_1$ and $x \is\pi T$ in $\Gamma_2$, $S \not\equiv T$.

\sectionwithtoc{Typing rules}

As we have done in the~\nameref{sec:dtlc} section, we formulate mutually
recursive rules that define how are the~well-formed contexts and
well-typed terms derived.

Contexts are identified within pseudo-contexts by the~judgement $\Gamma \vdash$,
defined by the~following rules:
\begin{mathpar}
  \inferrule*[right=Emp]
  { }
  {\diamond \vdash}

  \inferrule*[right=Ext]
  {\Gamma \vdash \\ 0\Gamma \vdash S \is0 \univ}
  {\Gamma, x \is\rho S \vdash}
\end{mathpar}
The~rule \ir{Emp} determines empty pseudo-context as valid. The~rule \ir{Ext}
extends the~context $\Gamma$ with a~new variable $x$ of type $S$, with usage
annotation $\rho$, which means that the~new context provides for a~$\rho$ uses
of $x$.

The~rules that derive a~valid typing judgment take the form:
\[
  \Gamma \vdash M \is\sigma S,
\]
where $\sigma$ is either $0$, or $1$. When $\sigma = 0$, we are indicating that
the~constructed term has no computational content; thus, all the~usage
annotations in the~context are necessarily also $0$. Stated as a~lemma, taken
from Atkey's Lemma 2.3:
\begin{lemma}[Zero needs nothing]\label{lem:zero_needs_nothing}
  If $\Gamma \vdash M \is0 S$, then $0\Gamma = \Gamma$.
\end{lemma}

All type formation rules bellow yield well-formedness judgments $\Gamma \vdash S
\is0 \univ$, meaning that type formation requires no computational resources.
When $\sigma = 1$, the~annotation indicates that we are constructing
computationally relevant data.

The~rules for variables and type conversion are:
\begin{mathpar}
  \inferrule*[right=Var]
  {0\Gamma_1, x \is\sigma S, 0\Gamma_2 \vdash}
  {0\Gamma_1, x \is\sigma S, 0\Gamma_2 \vdash x \is\sigma S}

  \inferrule*[right=Conv]
  {
    \Gamma \vdash M \is\sigma S \\\\
    0\Gamma \vdash S \equiv T
  }
  {\Gamma \vdash M \is\sigma T}

  \inferrule*[right=Univ]
  {0\Gamma \vdash}
  {0\Gamma \vdash \univ \is0 \univ}
\end{mathpar}
The~variable rule, \ir{Var}, selects a~single variable from the~context and all
the~other variables that do not take a~computational part in this judgment are
marked with $0$ usage. The~conversion rule, \ir{Conv}, is almost identical to
one in the~intuitionistic type theory \todo{[EXPLAIN]}, except that the~type
equality $S \equiv T$ is always judged in a~context with no resources.
The~type-in-type axiom, \ir{Univ}, is only available if $\sigma = 0$; we cannot
construct a~type that has a~runtime presence. In the~same vein, every type
formation rule in the~upcoming paragraphs, \ir{$\to$}, \ir{$\otimes$}, \ir{\1},
\ir{$\with$}, and \ir{$\top$}, forms a~type with usage $\sigma =0$ and thus
assumes a~zeroed-out context, consequence of \autoref{lem:zero_needs_nothing}.

The~\emph{dependent function types} \depq x \pi S T add an~annotation $\pi$,
which records how the~function will use its argument. The formation,
introduction, and elimination rules are:
\begin{mathpar}
  \inferrule*[right=$\to$]
  {0\Gamma \vdash S \is0 \univ \\ 0\Gamma, x \is0 S \vdash T \is 0 \univ}
  {0\Gamma \vdash \depq x \pi S T \is 0 \univ}

  \inferrule*[right=$\to$-I]
  {\Gamma, x \is{\sigma\pi} S \vdash M \is\sigma T}
  {\Gamma \vdash \lam x M \is\sigma \depq x \pi S T}

  \inferrule*[right=$\to$-E$_0$]
  {
    \Gamma \vdash M \is\sigma \depq x \pi S T \\\\
    0\Gamma \vdash N \is0 S \\
    \sigma\pi = 0
  }
  {\Gamma \vdash M \: N \is\sigma T}

  \inferrule*[right=$\to$-E$_1$]
  {
    \Gamma_1 \vdash M \is\sigma \depq x \pi S T \\\\
    \Gamma_2 \vdash N \is1 S
  }
  {\Gamma_1 + \sigma\pi\Gamma_2 \vdash M \: N \is\sigma T}
\end{mathpar}
In the~type formation rule \ir{$\to$}, the~annotation $\pi$ is not use used to
judge the~well-formedness of the~type, which follows from all type formation
rules being judged in a~context of $0$ usage. The~introduction rule uses $\pi$
to track how the~parameter $x$ is used and the~multiplication by $\sigma$
enforces the~zero-needs-nothing property of \autoref{lem:zero_needs_nothing};
namely, if we use the~function $0$-times, we also need $x$ $0$-times.
We have two distinct elimination rules, except for the~semiring $\{0\}$ where
the~rules are identical. The~rule \ir{$\to$-E$_0$} is applicable if $M \: N$ is
already being judged with $0$ usage ($\sigma = 0$), or if the~function does not
use its argument ($\pi = 0$). The~rule \ir{$\to$-E$_1$} is applicable otherwise,
but since the~term has a~computational usage ($\sigma \neq 0$), we must make
sure that the~$\sigma$ function uses have $\pi$ arguments' worth of resources,
hence the~$\sigma\pi$ scaling of the~context of the~function argument.

The~\emph{dependent multiplicative pair types} $(x \is \pi S) \otimes T$, like
the~function types, add an~annotation $\pi$ which records how many times
the~first element may be used. The formation, introduction, and elimination
rules are:
\begin{mathpar}
  \inferrule*[right=$\otimes$]
  {
    0\Gamma \vdash S \is 0 \univ \\
    0\Gamma, x \is 0 S \vdash T \is 0 \univ
  }
  {0\Gamma \vdash (x \is \pi S) \otimes T \is 0 \univ} \\

  \inferrule*[right=$\otimes$-I$_0$]
  {
    0\Gamma \vdash M \is 0 S \\
    \sigma\pi = 0 \\\\
    \Gamma \vdash N \is \sigma T[M/x]
  }
  {\Gamma \vdash \mpair M N \is \sigma (x \is \pi S) \otimes T}

  \inferrule*[right=$\otimes$-I$_1$]
  {
    \Gamma_1 \vdash M \is 1 S \\
    \Gamma_2 \vdash N \is \sigma T[M/x]
  }
  {
    \sigma\pi \Gamma_1 + \Gamma_2
    \vdash \mpair M N \is \sigma (x \is \pi S) \otimes T
  }

  \inferrule*[right=$\otimes$-E]
  {
    0\Gamma_1, z \is 0 (x \is \pi S) \otimes T \vdash O \is 0 \univ \\
    \Gamma_1 \vdash M \is \sigma (x \is \pi S) \otimes T \\
    \Gamma_2, x \is{\sigma \pi} S, y \is \sigma T
      \vdash N \is \sigma O[\mpair x y /z]
  }
  {
    \Gamma_1 + \Gamma_2 \vdash \textrm{let} \: \mpair x y \: \textrm{=}
    \: M \: \textrm{in} \: N \is \sigma O[M/z]
  }
\end{mathpar}
Dependent pair type being a~dual of dependent function type, the~introduction
rules \ir{$\otimes$-I$_i$} are analogous to the~elimination rules
\ir{$\to$-E$_i$}. The~rule \ir{$\otimes$-I$_0$} is applicable, if $\mpair M N$
is already being judged with $0$ usage ($\sigma = 0$), or if the~pair does not
use its first element computationally ($\pi = 0$). If $M$ has a~computational
presence, applying rule \ir{$\otimes$-I$_1$}, we have to scale $\Gamma_1$ by
$\sigma\pi$ to ensure that enough resources is available for $\sigma$ uses of
the~pair, each with $\pi$ uses of the~first element. The~elimination rule
ensures that both parts of the~pair are used.

The~\emph{multiplicative unit type} \1 has a~single value, \munit, and
the~following type formation, introduction, and elimination rules:
\begin{mathpar}
  \inferrule*[right=\1]
  {0\Gamma \vdash}
  {0\Gamma \vdash \1 \is 0 \univ}

  \inferrule*[right=\1-I]
  {0\Gamma \vdash}
  {0\Gamma \vdash \munit \is \sigma \1}

  \inferrule*[right=\1-E]
  {
    0\Gamma_1, u \is 0 \1 \vdash O \is 0 \univ \\\\
    \Gamma_1 \vdash M \is \sigma \1 \\
    \Gamma_2 \vdash N \is \sigma O[\munit/u]
  }
  {
    \Gamma_1 + \Gamma_2 \vdash \textrm{let} \: \munit \: \textrm{=}
    \: M \: \textrm{in} \: N \is \sigma O[M/u]
  }
\end{mathpar}
Constructing an~element of type \1 requires no resources. The~elimination of
elements of a~unit type must be done explicitly, via the~\emph{let} term, but is
otherwise straightforward.

The~\emph{dependent additive pair types} $(x \is{} S) \with T$ are without
the~semiring annotation on $x$~\todo{[EXPLAIN]}. The formation, introduction,
and elimination rules are:
\begin{mathpar}
  \inferrule*[right=$\with$]
  {0\Gamma \vdash S \is 0 \univ \\ 0\Gamma, x \is 0 S \vdash T \is 0 \univ}
  {0\Gamma \vdash (x \is{} S) \with T \is 0 \univ}

  \inferrule*[right=$\with$-I]
  {\Gamma \vdash M \is \sigma S \\ \Gamma \vdash N \is \sigma T[M/x]}
  {\Gamma \vdash \langle M, N \rangle \is \sigma (x \is{} S) \with T}

  \inferrule*[right=$\with$-Efst]
  {\Gamma \vdash M \is \sigma (x \is{} S) \with T}
  {\Gamma \vdash \fst M \is \sigma S}

  \inferrule*[right=$\with$-Esnd]
  {\Gamma \vdash M \is \sigma (x \is{} S) \with T}
  {\Gamma \vdash \snd M \is \sigma T[\fst M/x]}
\end{mathpar}

The~\emph{additive unit type}
The~formation and introduction are:
\begin{mathpar}
  \inferrule*[right=$\top$]
  {0\Gamma \vdash}
  {0\Gamma \vdash \top \is 0 \univ}

  \inferrule*[right=$\top$-I]
  {0\Gamma \vdash}
  {0\Gamma \vdash \aunit \is \sigma \top}
\end{mathpar}
Additive unit type has no elimination rule. We can never do anything interesting
with a~value of this type, except to incorporate it into a~larger value. We can
think about the~unit \aunit as a~\emph{garbage collection} for linear~resources.

