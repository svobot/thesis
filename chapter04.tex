\chapter{Quantitative Type Theory}

In this work, we extend the~Quantitative Type Theory (QTT) as first described by
McBride~\cite{mcbride_2016} and Atkey~\cite{atkey_2018}, with modifications by
Weirich~\cite{weirich_2020}.

Contexts are identified within precontexts by the~judgement $\Gamma \vdash$,
defined by the~following rules:
\begin{mathpar}
  \inferrule*[right=Emp]
  { }
  {\diamond \vdash}

  \inferrule*[right=Ext]
  {\Gamma \vdash \\ 0\Gamma \vdash S}
  {\Gamma, x \is\sigma S \vdash}
\end{mathpar}

\sectionwithtoc{Dependent function types}

\begin{mathpar}
  \inferrule*[right=Pi]
  {0\Gamma \vdash S \is0 \univ \\ 0\Gamma, x \is0 S \vdash T \is 0 \univ}
  {0\Gamma \vdash \depq x \pi S T \is 0 \univ}

  \inferrule*[right=Lam]
  {\Gamma, x \is{\sigma\pi} S \vdash M \is\sigma T}
  {\Gamma \vdash \lambda x . M \is\sigma \depq x \pi S T}

  \inferrule*[right=App$_0$]
  {
    \Gamma_1 \vdash M \is\sigma \depq x \pi S T \\
    \Gamma_2 \vdash N \is0 S \\
    0\Gamma_1 = 0\Gamma_2 \\
    \pi \cdot \sigma = 0
  }
  {\Gamma_1 \vdash M \: N \is\sigma T}

  \inferrule*[right=App]
  {
    \Gamma_1 \vdash M \is\sigma \depq x \pi S T \\
    \Gamma_2 \vdash N \is1 S \\
    0\Gamma_1 = 0\Gamma_2 \\
  }
  {\Gamma_1 + \pi\sigma\Gamma_2 \vdash M \: N \is\sigma T}
\end{mathpar}

\sectionwithtoc{Dependent multiplicative pair types}

\begin{mathpar}
  \inferrule*[right=$\otimes$]
  {
    0\Gamma \vdash S \is 0 \univ \\
    0\Gamma, x \is 0 S \vdash T \is 0 \univ
  }
  {0\Gamma \vdash (x \is \pi S) \otimes T \is 0 \univ}

  \inferrule*[right=$\otimes$-I$_0$]
  {
    \Gamma_1 \vdash M \is 0 S \\
    \Gamma_2 \vdash N \is \sigma T[M/x] \\
    0\Gamma_1 = 0\Gamma_2 \\
    \pi \cdot \sigma = 0
  }
  {\Gamma_2 \vdash (M, N) \is \sigma (x \is \pi S) \otimes T }

  \inferrule*[right=$\otimes$-I]
  {
    \Gamma_1 \vdash M \is 1 S \\
    \Gamma_2 \vdash N \is \sigma T[M/x] \\
    0\Gamma_1 = 0\Gamma_2
  }
  {\pi \sigma \Gamma_1 + \Gamma_2 \vdash (M, N) \is \sigma (x \is \pi S) \otimes T}

  \inferrule*[right=$\otimes$-E]
  {
    0\Gamma_1, p \is 0 (x \is \pi S) \otimes T \vdash O \is 0 \univ \\
    \Gamma_1 \vdash M \is \sigma (x \is \pi S) \otimes T \\
    \Gamma_2, x \is{\sigma \pi} S, y \is \sigma T \vdash N \is \sigma O[(x,y)/p] \\
    0\Gamma_1 = 0\Gamma_2
  }
  {
    \Gamma_1 + \Gamma_2 \vdash \textrm{let}_{O}(x, y) \: \textrm{=}
    \: M \: \textrm{in} \: N \is \sigma O[M/p]
  }
\end{mathpar}

\sectionwithtoc{Multiplicative unit type}

We use the~usual linear logic notation for the~multiplicative unit type, \1.
Although there should be no confusion, note the~difference between $1$, a~member of a~rig, and \1, a~multiplicative unit type.

\begin{mathpar}
  \inferrule*[right=\1]
  {0\Gamma \vdash}
  {0\Gamma \vdash \1 \is 0 \univ}

  \inferrule*[right=\1-I]
  {0\Gamma \vdash}
  {0\Gamma \vdash \munit \is \sigma \1}

  \inferrule*[right=\1-E]
  {
    0\Gamma_1, u \is 0 \1 \vdash O \is 0 \univ \\
    \Gamma_1 \vdash M \is \sigma \1 \\
    \Gamma_2 \vdash N \is \sigma O[\munit/u] \\
    0\Gamma_1 = 0\Gamma_2
  }
  {
    \Gamma_1 + \Gamma_2 \vdash \textrm{let}_{O} \: \munit \: \textrm{=}
    \: M \: \textrm{in} \: N \is \sigma O[M/u]
  }
\end{mathpar}

\sectionwithtoc{Dependent additive pair types}

\begin{mathpar}
  \inferrule*[right=$\with$]
  {0\Gamma \vdash S \is 0 \univ \\ 0\Gamma, x \is 0 S \vdash T \is 0 \univ}
  {0\Gamma \vdash (x \is{} S) \with T \is 0 \univ}

  \inferrule*[right=$\with$-I]
  {\Gamma \vdash M \is \sigma S \\ \Gamma \vdash N \is \sigma T[M/x]}
  {\Gamma \vdash \langle M, N \rangle \is \sigma (x \is{} S) \with T}

  \inferrule*[right=$\with$-Efst]
  {\Gamma \vdash M \is \sigma (x \is{} S) \with T}
  {\Gamma \vdash \textrm{fst} \: M \is \sigma S}

  \inferrule*[right=$\with$-Esnd]
  {\Gamma \vdash M \is \sigma (x \is{} S) \with T}
  {\Gamma \vdash \textrm{snd} \: M \is \sigma T[\textrm{fst}\: M /x]}
\end{mathpar}

\sectionwithtoc{Additive unit type}

\begin{mathpar}
  \inferrule*[right=$\top$]
  {0\Gamma \vdash}
  {0\Gamma \vdash \top \is 0 \univ}

  \inferrule*[right=$\top$-I]
  {0\Gamma \vdash}
  {0\Gamma \vdash \aunit \is \sigma \top}

\end{mathpar}

