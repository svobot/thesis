\chapter{Lambda Calculus}\label{cha:lambdacalculus}

The~term \emph{\lc} has come to refer to a~collection of several formal systems.
They are all based on a~notation that was introduced by Alonzo Church in
the~1930s. \citet{church_1932} originally proposed the~\lc as a~way to define
notations for logical formulas while trying to formulate a~foundation of
mathematics based on functions. Church's students \citet{kleene_rosser_1935}
have however shown that this system was inconsistent. The~\lc thus found
a~different application, pointed out by Church from the~start, in formally
describing the~informal notion of effective computability. The~formal definition
was required to show the~undecidability of Hilbert's \emph{Entscheidungsproblem}.

The~various extensions of Church's simplest, \emph{pure}, \lc usually introduce
some forms of additional grammatical structure, depending on its intended use.
Most of the~systems have some syntactic restrictions built-in, like
type-restrictions. We will examine the~types in the~next chapter, but first we
will describe the~pure \lc.

As a~motivation for the~$\lambda$-notation, consider the~mathematical
expression $x + y$. We can think of this as either defining a~function $f$ of
$x$ or a~function $g$ of $y$;
\begin{align*}
  f(x) = x + y,&  &  &g(y) = x + y,
\intertext{or}
  f : \;\; x \mapsto x + y,&  &  &g : \;\; y \mapsto x + y.
\end{align*}
With more complex terms, we can imagine the~need for a~notation giving $f$ and
$g$ different names in some systematic way.

Church came up with a~systematic way of constructing, for each expression
involving $x$, a~notation for the~corresponding function of $x$, and similarly
for $y$, etc. He introduced $\lambda$ as an~auxiliary symbol and wrote
the~functions as
\begin{align*}
  f = \lam{x}{x + y},&  &  &g = \lam{y}{x + y}.
\intertext{In this example, applications}
  f(0) = 0 + y,&  &  &g(1) = x + 1
\intertext{become}
  (\lam{x}{x + y})(0) = 0 + y,&  &  &(\lam{y}{x + y})(1) = x + 1.
\end{align*}
We can consider extending the~notation to functions of more than one variable.
For example, the~expression $x + y$ determines two functions $h$ and $k$ of two
variables
\begin{align*}
  h(x,y) = x + y,&  &  &k(y,x) = x + y.
\intertext{These can be denoted by}
  h = \lam{xy}{x + y},&  &  &k = \lam{yx}{x + y}.
\end{align*}
Instead, we will use functions of one variable to \emph{represent} all other
functions as well. For example, instead of the~function $h$, consider
the~function $h^\star$ defined by
\[
  h^\star = \lam x (\lam y {x + y}).
\]
For each value $a$, we have
\begin{align*}
  h^\star(a) &= \lam y {a + y};
\intertext{hence for each pair of values $a, b$,}
  (h^\star(a))(b) &= (\lam y {a + y})(b) \\
                  &= a + b \\
                  &= h(a, b).
\end{align*}
The~use of $h^\star$ instead of $h$ is usually called \emph{currying}, after
the~logician Haskell Curry. As is usually the~case, from here on we will
restrict ourselves to the~use of curried functions.

Now that we have seen the~$\lambda$-notation in an~informal context, let us
construct a~formal system of \lc.
\begin{definition}\label{def:lambda_calculus}
  Assume that we are given an~infinite sequence of expressions $\mathbf{v}_0,
  \mathbf{v}_1, \mathbf{v}_2, \dots$ called \emph{variables}, then the~set of
  expressions called $\lambda$-terms is defined inductively as follows:
  \begin{enumerate}
    \item all variables are $\lambda$-terms (called \emph{atoms});
    \item if $M$ and $N$ are $\lambda$-terms, then $(M N)$ is a~$\lambda$-term
      (called \emph{application});
    \item if $M$ is a~$\lambda$-term and $x$ is a~variable, then $(\lam x M)$ is
      a~$\lambda$-term (called an~\emph{abstraction}).
  \end{enumerate}
\end{definition}
Like in the~definition, capital letters will denote arbitrary $\lambda$-terms.
Letters $x, y, z, u, v, w$ will denote variables. Application is
left-associative and we will omit parentheses accordingly. For example,
$M N P Q$ denotes the~term $(((M N) P) Q)$. Other abbreviations are
\begin{align*}
  &{\lam x {P Q}}  &  &\text{for}  &  &(\lam x {(P Q)}) \\
  &{\lam {x_1 x_2 \dots x_n} M}  &  &\text{for}  &
    &(\lam {x_1} {(\lam {x_2} {(\dots(\lam {x_n} M) \dots)})}).
\end{align*}
\emph{Syntactic identity} of terms, written as $M \equiv N$, means that
the~terms $M$ and $N$ are exactly the~same.

\sectionwithtoc{Substitution}

Our goal is to be able to formulate a~procedure for calculating with terms in
accordance with their informal meaning. Before we formalise this, we need to
look at how to substitute terms for variables.

\begin{definition}
  Given $\lambda$-terms $P$ and $Q$, we define a~relation \emph{$P$ occurs in
  $Q$} by induction on $Q$:
  \begin{enumerate}
    \item $P$ occurs in $P$;
    \item if $P$ occurs in $M$ or in $N$, then $P$ occurs in $(M N)$;
    \item if $P$ occurs in $M$ or $P \equiv x$, then $P$ occurs in $(\lam x M)$.
  \end{enumerate}
\end{definition}

Note that, for example, in the~term $(\lam x (x y)) (x y)$, there are two
occurrences of $(x y)$ and three occurrences of $x$.

\begin{definition}
  For a~particular occurrence of $\lam x M$ in a~term $P$, the~occurrence of $M$
  is called the~\emph{scope} of the~occurrence of $\lambda x$ on the~left.
\end{definition}

\begin{definition}[Free and bound variables]
  A~variable $x$ in a~term $P$ is called
  \begin{itemize}
    \item \emph{bound} if its occurrence is in the~scope of a~$\lambda x$ in $P$,
    \item \emph{bound and binding}, if and only if it is the~$x$ in $\lambda x$,
    \item \emph{free} otherwise.
  \end{itemize}
  Free variables form a~set, called $\fv P$.
\end{definition}

For example, in the~term $P \equiv xy(\lam {zv} {zy})w$ the~leftmost $x, y$ and
the~rightmost $w$ are all free variables, the~$z$ and $v$ in $\lam {zv} {}$ are
bound and binding, the~rightmost $z$ is bound, and the~rightmost $y$ is free;
hence
\[
  \fv P = \{x, y, w\} .
\]

\begin{definition}[Substitution]\label{def:substitution}
  Given terms $M$ and $N$, the~result of substituting $N$ for every occurrence
  of the~free variable $x$, written $[N/x]M$, is defined by induction on $M$:
  \begin{enumerate}
    \item $[N/x]x  \equiv N$;
    \item $[N/x]a  \equiv a$  \hfill for all atoms $a \not\equiv x$;
    \item $[N/x](P Q)  \equiv ([N/x]P [N/x]Q)$;
    \item $[N/x](\lam x P)  \equiv \lam x P$;
    \item $[N/x](\lam y P)  \equiv \lam y P$ \hfill if $x \notin \fv P$;
    \item $[N/x](\lam y P)  \equiv \lam y {[N/x]P}$
      \hfill if $x \in \fv P$ and $y \notin \fv N$;
    \item \label{def:substitution:g_item} $[N/x](\lam y P)
      \equiv \lam z {[N/x][z/y]P}$ \hfill if $x \in \fv P$ and $y \in \fv N$,
  \end{enumerate}
  where $y \not\equiv x$ and $z$ is chosen as the~first variable $\notin \fv N
  \cup \fv P$.
\end{definition}

Note that the~case \ref{def:substitution:g_item} involves replacing the~concrete
bound variables such that they do not clash with the~free variables in the~term
that is being substituted in. In fact, the~choice of any particular bound
variable is entirely arbitrary and it can be exchanged for another available
variable. The~following definition formalises this idea.

\begin{definition}
  Given a~term $P$ containing an~occurrence of $\lam x M$ and a~variable $y
  \notin \fv M$, the~act of replacing this $\lam x M$ by
  \[
    \lam y {[y/x]M}
  \]
  is called an~\emph{$\alpha$-conversion}, or \emph{a~change of bound variable}
  in $P$.

  The~reflexive-transitive closure of the~$\alpha$-conversion relation is
  the~\emph{$\alpha$-converts to} relation, or
  \[
    P \equiv_\alpha Q.
  \]
\end{definition}

Reflexive-transitive closure of a~relation in this case means, that we are able
to find a~sequence (possibly empty) of $\alpha$-conversions, that transforms
the~term $P$ into $Q$.

Having looked at the~way we can combine $\lambda$-terms via substitution, we can
turn our focus to the~calculation procedure that lies at the~heart of \lc and
gives it its power.

\sectionwithtoc{\ensuremath{\beta}-reduction}

As the~name in the~\autoref{def:lambda_calculus} suggests, a~term of
the~form $(\lam x M) N$ represents an~application of an~operator $\lam x M$ to
an~argument $N$. The~resulting value of such application is the~\emph{simplified}
term $[N/x]M$. This simplification process is captured in the~following
definition.

\begin{definition}
  Any term of the~form
  \[
    (\lam x M) N
  \]
  is called a~\emph{$\beta$-redex} and the~corresponding term
  \[
    [N/x]M
  \]
  is called \emph{contractum}. If and only if a~term $P$ contains an~occurrence
  of $(\lam x M) N$ and we replace that occurrence by $[N/x]M$, and the~result
  is $P'$, we say we have \emph{contracted} the~redex-occurrence in $P$, and $P$
  \emph{$\beta$-contracts} to $P'$, or
  \[
    P \; \triangleright_{1\beta} \; P'.
  \]
  The~reflexive-transitive closure of the~$\beta$-contracts relation is
  the~\emph{$\beta$-reduces to} relation, or
  \[
    P \triangleright_\beta Q.
  \]
\end{definition}

Informally speaking, the~$\lambda$-terms are called \emph{programs}, which we
can then think of as \emph{executing}, or \emph{running}, instead of
$\beta$-reducing.

