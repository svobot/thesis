\chapwithtoc{Conclusion}

\coloredlettrine{O}{ur goal} was to build an~interpreter for a~language based on
the~Quantitative Type Theory. We have used bidirectional versions of the~typing
rules to decrease the~type annotation burden on the~user. The~implementation
is based on the~\lc with QTT rules that we have described in the~text. Compared
to previous works on this topic, we have extended the~type system with
the~additive pair and unit types.

Since distinguishing between the~semantics of the~multiplicative and additive
pairs was an~important part of this work, we have implemented highlighting of
the~interpreter output, which differentiates the~two types at first glance.

Currently, the~interpreter implementation makes only one semiring with one order
available to the~user; additional semiring options could be added. Janus is also
unable to parametrise over the~semiring elements, adding this feature could
increase code reuse.
