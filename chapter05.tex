\chapter{Implementation}\label{cha:implementation}

\coloredlettrine{Q}{uantitative Type} Theory has already found practical
application in the~programming language Idris~2 by \citet{brady_2021}. Idris~2
aims to be a~full-scale programming language where the~programmer can experiment
with designs that QTT enables.

The~main focus of this work was to crate an~interpreter of a~\lc with
quantitative types. The~implementation is written in Haskell\citep{haskell_2010}
and it is based on the~dependently typed \lc interpreter \emph{LambdaPi} created
by \citet{loh_et_al_2010}. We have created a~simple programming language, which
we call \emph{Janus}, to use in our interpreter. Later in this chapter, we
detail the~syntax of the~language, as well as how to use the~interpreter.

Typing rules in the~previous chapter are presented ``declaratively,'' echoing
approach taken by \citet{atkey_2018}, which is well suited for semantical
investigation of the~theory; \citet{mcbride_2016} presents the~system
differently, using rules in the~\emph{bidirectional} form. This form is
convenient when implementing the~typing algorithm, so our next step is to covert
the~rules from the~previous chapter into their bidirectional form.

\sectionwithtoc{Bidirectional typing}\label{sec:bidir_typing}

Bidirectional typing splits the~typing judgment $\Gamma \vdash M \is{\sigma} S$
into two judgments:
\begin{align*}
  \text{\emph{type checking}}\qquad &\Gamma \vdash M \chck{\sigma} S \\
  \text{\emph{type synthesis}}\qquad &\Gamma \vdash M \syn{\sigma} S
\end{align*}
Type synthesis is also called \emph{type inference} and the~term $M$ is
the~\emph{subject} of these judgments. The~difference is that in type checking
the~type $S$ is an~input and the~derivation proves that the~subject can indeed
be of type $S$; whereas, in type synthesis the~type $S$ is an~output.

It is often the~case that checking a~given solution to a~problem is easier than
finding, or synthesising a~solution. In typing, this holds as well: synthesising
a~type may be convenient for the~programmer, but undecidable for the~algorithm,
so at some points in the~program the~programmer has to guide the~typing
algorithm by providing the~types. Bidirectional typing systems can alternate
between treating types as inputs or outputs and so the~resulting language can
take advantage of various typing features while demanding only a~modest volume
of type annotations from the~programmer. We will explain bidirectional typing
concepts that are relevant to this work, but interested reader is encouraged to
consult the~survey on the~topic provided by \citet{dunfield_krishnaswami_2019}
for further details.

In bidirectional typing, we need to decide which terms can have their type
synthesised and which terms can only have it checked. Notice that we are saying
that if we can synthesise a~type from a~term, we can then also take the~result
and check the~term with it. In the~rest of this section, we use $I$ to denote
terms from which its type can be synthesised, or inferred, and we use $C$ for
terms that can have its type checked, which is every term.

The~following terms can have its type synthesised:
\begin{enumerate}
  \item type annotation
    \[
      C_1 \is{} C_2
    \]
    lets us create a~type-synthesising term from an~arbitrary term $C_1$ by
    attaching the~type $C_2$ to it. All type annotations in these terms are
    appended following a~colon. This is closer to the~actual syntax of our
    interpreter where we have been constrained by what can be easily typed on
    a~keyboard;
  \item variable
    \[
      x
    \]
    has its type specified in the~typing context;
  \item function application
    \[
      I \: C
    \]
    combines a~type-synthesising term which evaluates to a~function and
    a~type-checkable term that provides the~function argument;
  \item dependent multiplicative pair eliminator
    \[
      \letpair {} [z] x y I {C_1} \is{} C_2
    \]
    takes a~type-synthesising term that produces the~pair and type-checkable
    terms $C_1$, which uses the~pair and $C_2$, providing the~type annotation of
    $C_1$ and again written following a~colon. As before, $x$ and $y$ bind
    the~two elements of the~pair in $C_1$ and $z$ binds the~whole pair in $C_2$.
    Hence, we write
    \[
      \letpair {} [z] x y M N \is{} O \qquad \text{instead of} \qquad
        \letpair {(x \is\pi S) \otimes T} [z] x y M [O] N,
    \]
    for some types $S$ and $T$;
  \item multiplicative unit eliminator
    \[
      \letu* [x] I {C_1} \is{} C_2
    \]
    also moves the~type annotation $C_2$ of the~term $C_1$ after the~colon;
    hence, we write
    \[
      \letu* [x] M N \is{} O \qquad \text{instead of} \qquad
        \letu [x] M [O] N;
    \]
  \item additive pair eliminators
    \begin{mathpar}
      \fst I \and \snd I
    \end{mathpar}
    takes a~type-synthesising term that produces the~pair.
\end{enumerate}

Terms that can have its type checked are:
\begin{enumerate}
  \item every type-synthesising term
    \[
      I
    \]
    is also type-checkable, since we can take the~result of the~type synthesis
    and check the~term with it;
  \item atomic types
    \begin{mathpar}
      \univ \and \1 \and \top
    \end{mathpar}
    as well as the~units themselves,
    \begin{mathpar}
      \munit \and \aunit
    \end{mathpar}
    are type-checkable;
  \item function abstraction term
    \[
      \lam x C
    \]
    has no explicit typing annotation of its parameter, instead we have to
    specify the~type somewhere else. For example, we usually annotate the~whole
    function; if $M$ has type $T$, we can write
    \[
      \lam x M \is{} \depq x \pi S T \qquad \text{instead of} \qquad
        \lam {x \is\pi S} M;
    \]
  \item terms for the~formation of function and pair types
    \begin{mathpar}
      \depq x \pi {C_1}{C_2} \and (x \is \pi C_1) \otimes C_2 \and
        (x \is{} C_2) \with C_1
    \end{mathpar}
    and terms for the~introduction of pairs
    \begin{mathpar}
      \mpair{C_1}{C_2} \and \apair{C_1}{C_2}
    \end{mathpar}
    combine type-checkable terms, but are otherwise unchanged from their
    counterparts in the~previous chapter.
\end{enumerate}

Whether a~given term is type-synthesising or only type-checkable is determined
by the~role it plays in the~bidirectional typing rules. Our definition follows
the~\emph{Pfenning recipe}~\citep{dunfield_krishnaswami_2019}:
\begin{quote}
  If the~rule is an~\emph{introduction} rule, make the~principal judgment
  \emph{checking} and if the~rule is an~\emph{elimination} rule, make
  the~principal judgment \emph{synthesising}.
\end{quote}
A~\emph{principal judgment} is the~judgment containing the~logical connective
that is being introduced or eliminated; in our introduction rules the~principal
judgment is always the~conclusion and in the~elimination rules it is the~first
premise. Thus, terms that are the~subject of type synthesis judgments are
type-synthesising, otherwise they are only type-checkable.

The~bidirectional typing rules are, except for the~directionality, identical to
the~rules in the~previous chapter. They are written in full in
the~\autoref{cha:bidir_rules}. The~only new rule we add synthesises the~type of
term $M \is{} S$:
\begin{mathpar}
  \inferrule*[right=Ann\syn{}]
  {\Gamma \vdash M \chck{\sigma} S}
  {\Gamma \vdash M \is{} S \syn{\sigma} S}
\end{mathpar}

The~bidirectional typing rules are almost completely \emph{syntax-directed};
that is, the~derivation rule which can be used is determined by the~syntax of
the~term. At any point, there is only one rule whose subject in the~conclusion
judgment matches the~term. There are two exceptions to this: first, when
constructing a~multiplicative pair, we first check if $\sigma\pi = 0$ and use
the~rule \ir{$\otimes$-I$_0$\chck{}}, otherwise we use
\ir{$\otimes$-I$_1$\chck{}}; and second, when applying a~function, we synthesise
the~type of the~function first and then check if $\sigma\pi = 0$ as well,
using the~rule \ir{$\to$-E$_0$\syn{}} if true and \ir{$\to$-E$_1$\syn{}}
otherwise.

\sectionwithtoc{Resource accounting}\label{sec:usage}

To be able to translate the~typing rules into an~efficient algorithm, we need to
deal with one last thing. Currently, it seems as if we have to guess how to
split the~resources in a~given context when deriving multiple premises. For
example, in the~rule
\begin{mathpar}
    \inferrule*[right=$\to$-E$_1$\syn{}]
  {
    \Gamma_1 \vdash M \syn\sigma \depq x \pi S T \\
    \Gamma_2 \vdash N \chck1 S
  }
  {\Gamma_1 + \sigma\pi\Gamma_2 \vdash M \: N \syn\sigma T}
\end{mathpar}
we are given the~context $\Gamma_1 + \sigma\pi\Gamma_2$ and we need to decide
how it gets split into $\Gamma_1$ and $\Gamma_2$ to be used in the~premises. We
remove the~guessing from the~algorithm by adding outputs to our typing
judgments: both type checking and type synthesis output a~\emph{usage}, a~record
of how much of each variable is the~subject term consuming.

\begin{definition}
  If $V$ is a~set of variables and $\mathcal{R}$ is a~semiring, then
  \emph{usage} is a~subset of the~Cartesian product $V \times \mathcal{R}$, such
  that every variable is unique.
\end{definition}

We will write usage as $\{x \: \pi; \: y \: \rho; \dots \}$ instead of
$\{(x, \pi), (y, \rho), \dots\}$ to not confuse it with the~multiplicative pair
notation.

Usage accumulates the~resource consumption from throughout the~term and
the~typing algorithm uses it to check that an~appropriate amount of every
resource is available. For \emph{global variables}, i.e. variables that are free
in the~whole term, this check is done at the~end of the~typing algorithm and for
the~\emph{local variables}, i.e. variables that have a~binding occurrence
somewhere in the~term, the~check is done after accumulating its usage from its
scope. Thus, returning to the~example of rule \ir{$\to$-E$_1$\syn{}}, there is
no need to know how to split $\Gamma_1 + \sigma\pi\Gamma_2$ a~priori, instead
the~premises produce usages $\Psi$ from $M$ and $\Omega$ from $N$, which get
combined into $\Psi + \Omega$ and returned as the~usage of $M \: N$.
The~addition on usages works analogously to contexts.

We will not extend the~formalisation of the~bidirectional rules to include
\emph{usage}, instead we will go through an~example and use it to illustrate how
usage works. Consider the~judgment:
\[
  \Gamma \vdash ((\lam y {f \: x \: y}) \is{} \depq y 1 {(z \is1 T)\otimes T} S)
    \: \mpair x x \syn1 S
\]
To find out whether the~variables in $\Gamma$ have the~quantities needed to
successfully synthesise $S$, we collect the~usage from the~term and then check
it against the~context. Here is part of the~derivation:
\begin{mathpar}
  \inferrule*[Right=$\to$-E$_1$\syn{}]
  {
    \inferrule*[right=Ann\syn{}]
    {
      \inferrule*
      { \vdots }
      {0\Gamma \vdash M \chck0 \univ} \\
      \inferrule*[Right=$\to$-I\chck{}]
      {
        \inferrule*[Right=Inf\chck{}]
        {
          \inferrule*[Right=$\to$-E$_1$\syn{}]
          {
            \inferrule*
            { \vdots }
            {\Gamma'_1 \vdash f \: x \syn1 M} \\
            \inferrule*
            { \vdots }
            {\Gamma'_2 \vdash y \chck1 (z \is1 T)\otimes T}
          }
          {\Gamma', y \is1 (z \is1 T)\otimes T \vdash f \: x \: y \syn1 S}
        }
        {\Gamma', y \is1 (z \is1 T)\otimes T \vdash f \: x \: y \chck1 S}
      }
      {\Gamma' \vdash \lam y {f \: x \: y} \chck1 M}
    }
    {
      \Gamma' \vdash
      (\lam y {f \: x \: y})\is{} M \syn1 M
    } \\
    \inferrule*
    { \vdots }
    {\mathcal{J} \;}
  }
  {\Gamma \vdash ((\lam y {f \: x \: y})\is{} M) \: \mpair x x \syn1 S}
\end{mathpar}
where, to make the~terms more readable, $M$ stands for
$\depq y 1 {(z \is1 T)\otimes T} S$. Since the~term is computationally relevant,
given the~\syn1, we start the~derivation with rule \ir{$\to$-E$_1$\syn{}}, which
first expects us to synthesise the~type of the~annotated λ-abstraction. This in
turn demands the~use of rules \ir{Ann\syn{}} and \ir{$\to$-I\chck{}}. The~first
premise of \ir{Ann\syn{}} checks that $M$ is a~type and produces a~usage which,
since we are in the~erased part of the~theory (\chck0), has to be empty.
The~function introduction rule adds variable $y$ to the~context; it is
the~\emph{local variable}, so after collecting the~usage from $f \: x \: y$, we
check the~consumed quantity of $y$: the~term $f \: x$ produces the~usage
$\{f \:1; \: x \: 1\}$ and $y$ produces $\{y \: 1\}$, which added together gives
$\{f \:1; \: x \: 1; \: y \: 1\}$. This is a~valid usage, since our context
contains $y$ with quantity 1. Now $y$ goes out of scope again and so it is
removed from the~usage; thus, the~whole term $\lam y {f \: x \: y}$ uses
$\{f \:1; \: x \: 1\}$. This, added together with the~empty usage from checking
the~type of $M$, gives that $(\lam y {f \: x \: y})\is{} M$ also uses
$\{f \:1; \: x \: 1\}$. Judgment $\mathcal{J}$ uses $\{x \: \omega\}$, since
the~$x$ is used twice. Finally, the~whole term uses
$\{f \:1; \: x \: 1\} + \{x \: \omega\} = \{f \:1; \: x \: \omega\}$. This means
that $\Gamma$ has to contain
\[
  f \is1 \depq x 1 T {\depq y 1 {(z \is1 T)\otimes T} S}, \: x \is\omega T
\]
for the~typing to succeed. Note that in the~derivation we omit some premises of
the~\ir{Inf\chck{}} rule for brevity.

\sectionwithtoc{Janus}

This work is accompanied by an~interpreter of a~\lc with quantitative types.
The~interpreter uses the~\emph{Janus} language to express terms in that \lc.
This section details syntax of the~language. The~listings use monospaced font
when quoting the~Janus language, but use proportional cursive for metavariables.

The~interpreter provides a~read-evaluate-print loop (REPL) for a~user to
interact with. Each iteration of this loop expects the~input to either be
a~\emph{command} or a~\emph{statement}. Throughout its run, the~interpreter
maintains a~state with the~results of previously evaluated statements, which are
then available in the~contexts of subsequent statements. Commands are identified
by a~leading colon and some expect arguments. Arguments follow the~command,
separated by whitespace. Our interpreter provides the~following commands:
\begin{itemize}
  \item \emph{Load} command (\janus{:load}) expects a~file path as an argument.
    It reads the~file and evaluates its contents. A~valid file contains
    a~sequence of Janus statements which are evaluated consecutively. For
    example, to load a~file \emph{library.jns} in the~current directory, user
    writes:
    \begin{center}
      \janus{:load library.jns}
    \end{center}
  \item \emph{Browse} command (\janus{:browse}) lists all variables and their
    types that the~state of the~interpreter currently tracks.
  \item \emph{Type} command (\janus{:type}) expects as an~argument
    a~type-synthesising Janus term and it synthesises the~type.
  \item \emph{Quit} command (\janus{:quit}) exits the~interpreter.
  \item \emph{Help} command (\janus{:help} or \janus{:?}) shows a~short
    description of the~interpreter's features.
\end{itemize}
Any prefix of a~command is also recognised: the~user can, for example, write
\janus{:l library.jns}, or \janus{:q}.

Before we look at the~statements, we will define the~syntax of terms in
the~Janus language. These terms are a~straightforward analogues of
the~terms from the~\nameref{sec:bidir_typing} section. The~following table
details how to translate the~terms from the~mathematical notation into
the~syntax our interpreter understands; each term has an~ASCII variant that
is easily typed on a~keyboard and a~Unicode variant that closer corresponds to
the~mathematical notation.

Parts of the~syntax are colourised, which is done to highlight the~differences
between the~multiplicative and additive pairs: constructors, eliminators, and
types for multiplicative pairs and units are printed in
blue\colorrect{DodgerBlue2}; and similarly, syntax of additive terms and types
is printed in green\colorrect{Green4}. Every resource semiring annotation is
printed in magenta\colorrect{DeepPink3}. These colours are produced using
the~ANSI escape codes and so should be available in all modern terminal
emulators. Additionally, binding occurrences of variables are printed in bold.

\begin{center}
\begin{tabular*}{\textwidth}{@{\extracolsep{\fill} } c c c }
  Abstract syntax & ASCII variant & Unicode variant \\
  \hline \hline
  \multicolumn{3}{ c }{Function} \\
  \hline
  \multirow{2}{*}{$\depq x \pi S T$}
    & \janus{(\quant{$\pi$} $x$ : $S$) -> $T$}
    & \janus{\vdep{$\pi$}{$x$}{$S$}{$T$}} \\
    & \janus{forall (\quant{$\pi$} $x$ : $S$) . $T$}
    & \janus{∀ (\quant{$\pi$} $x$ : $S$) . $T$} \\
  $\lam x M$ & \janus{\textbackslash$x$. $M$} & \janus{\vlam{$x$}{$M$}} \\
  $M \: N$   & \janus{{$M$} {$N$}}            & --- \\

  \hline \hline
  \multicolumn{3}{ c }{Multiplicative pair} \\
  \hline
  $(x \is \pi S) \otimes T$
    & \janus{\mult(\quant{$\pi$} $x$ : $S$\mult) \mult* $T$}
    & \janus{\vmpairt{$\pi$}{$x$}{$S$}{$T$}} \\
  $\mpair M N$ & \janus{\vmpair{$M$}{$N$}} & --- \\
  $\letpair {} [z] x y M N \is{} O$ &
    \multicolumn{2}{c}{\janus{\vmpairlet{$z$}{$x$}{$y$}{$M$}{$N$}{$O$}}} \\

  \hline \hline
  \multicolumn{3}{ c }{Multiplicative unit} \\
  \hline
  \1     & \janus{\mult{I}} & \janus{\vmunitt} \\
  \munit & \janus{\vmunit}  & ---              \\
  $\textrm{let} \: x @ \munit \: \textrm{=} \: M \: \textrm{in} \: N \is{} O$ &
    \multicolumn{2}{c}{\janus{\vmunitlet{$x$}{$M$}{$N$}{$O$}}} \\

  \hline \hline
  \multicolumn{3}{ c }{Additive pair} \\
  \hline
  $(x \is{} S) \with T$ & \janus{\vapairt{$x$}{$S$}{$T$}} & --- \\
  $\apair M N$ & \janus{\add<$M$\add, $N$\add>} & \janus{\vapair{$M$}{$N$}} \\
  $\fst M$     & \janus{\vapairfst {$M$}}       & --- \\
  $\snd M$     & \janus{\vapairsnd {$M$}}       & --- \\

  \hline \hline
  \multicolumn{3}{ c }{Additive unit} \\
  \hline
  $\top$ & \janus{\add{T}}  & \janus{\vaunitt} \\
  \aunit & \janus{\add{<>}} & \janus{\vaunit} \\

  \hline \hline
  \multicolumn{3}{ c }{Universe} \\
  \hline
  $\univ$ & \janus{U} & \janus{\vuniv} \\

  \hline \hline
  \multicolumn{3}{ c }{Type annotation} \\
  \hline
  $M \is{} N$ & \janus{$M$ : $N$} & --- \\

  \hline \hline
  \multicolumn{3}{ c }{Zero-One-Many semiring} \\
  \hline
  $0$      & \janus{\vzero}    & --- \\
  $1$      & \janus{\vone}     & --- \\
  $\omega$ & \janus{\quant{w}} & \janus{\vmany} \\
\end{tabular*}
\end{center}

Binding variables that have no other occurrence in the~term are often written as
\janus{\_}, following a~convention from the~Haskell language; for example,
these two terms are equivalent:
\begin{center}
  \janus{(\vlam{x y}{y}) : \vdep{0}{x}{U}{\vdep{1}{y}{x}{x}}} \\ \vspace{0.6em}
  \janus{(\vlam{\_ y}{y}) : \vdep{0}{x}{U}{\vdep{1}{\_}{x}{x}}}
\end{center}

The~dependent function type can be written in two ways, either using an~arrow,
or a~universal quantifier. The~second form is useful for describing types of
higher order functions, because we can list all the~binding variables without
any interspersing syntax; meaning, in Janus, the~term
$\depq{x_1}{\pi_1}{S_1}{\ldots \to \depq{x_n}{\pi_n}{S_n}{T}}$ can be written in
two ways:
\begin{center}
  \janus{\vdep{$\pi_1$}{$x_1$}{$S_1$}{$\ldots$} →
    \vdep{$\pi_n$}{$x_n$}{$S_n$}{$T$}} \\ \vspace{0.6em}
  \janus{∀ (\quant{$\pi_1$} $x_1$ : $S_1$) $\ldots$
    (\quant{$\pi_n$} $x_n$ : $S_n$) . $T$}
\end{center}

As in the~mathematical notation, function abstractions can be abbreviated to
bind multiple variables at the~same time; so we can write
\begin{center}
  \janus{\vlam{x y z}{x}} \hspace{3em} instead of \hspace{3em}
  \janus{\vlam{x}{\vlam{y}{\vlam{z}{x}}}}
\end{center}

Another convenience abbreviation is available: the~semiring annotations can be
omitted; interpreter then uses $\omega$, hence the~following terms are
equivalent:
\begin{center}
  \janus{\vdep{\vmany}{\_}{$S$}{\vmpairt{\vmany}{\_}{$T$}{$S$}}} \hspace{6em}
  \janus{(\_ : $S$) → \mult(\_ : $T$\mult{) ⊗} $S$}
\end{center}
This is also true for the~semiring annotations anywhere in the~Janus statements
below.

Janus terms form the~\emph{statements} that our interpreter understands. REPL
evaluates the~terms in these statements and, if needed, modifies the~running
interpreter state with the~results. The~new state is then used when evaluating
subsequent statements. So, next to the~commands listed above, our interpreter
recognises three kinds of statements:

First, there is the~\emph{assume} statement, which adds variables to
the~interpreter state. The~state contains a~running context, which is used in
typing judgments that are derived during the~run of the~interpreter. This
context gets extended with the~variables specified using the~assume statement.
The~syntax of the~statement is:
\begin{center}
  \janus{assume (\quant{$\pi_1$} $x_1$ : $S_1$) $\dots$\
  (\quant{$\pi_n$} $x_n$ : $S_n$)}
\end{center}
where $n \geq 1$, $\pi_i$ is a~semiring element, $x_i$ is a~variable, and $S_i$
is a~type-checkable term. The~interpreter consecutively, for every $i$, tries
deriving the~judgment
\[
  0\Gamma_i \vdash S_i \is{} \univ \syn0 \univ
\]
where $\Gamma_i$ is the~state context, and if it successfully synthesises
the~type, it adds $x_i \is{\pi_i} S_i$ to the~context; that is, it replaces
the~context $\Gamma_i$ in the~state with the~extended
\[
  \Gamma_i, x_i \is{\pi_i} S_i
\]
for the~subsequent judgments. For example, the~statement
\begin{center}
  \janus{assume (\quant0 $x$ : U) ($y$ : \vdep{1}{\_}{$x$}{$x$})}
\end{center}
extends the~interpreter state context with $x \is0 \univ, y \is\omega
\depq z 1 x x$. Note that $x$ is already in the~state context, when judging
$0\Gamma \vdash \depq z 1 x x \is{} \univ \syn0 \univ$.

Second, the~\emph{let} statement adds to the~interpreter state a~variable
together with its value. The~statement takes the~form
\begin{center}
  \janus{let \quant{$\pi$} $x$ = $M$}
\end{center}
where $\pi$ is a~semiring element, $x$ is a~variable, and $M$ is
a~type-synthesising term. If the~judgment $\Gamma \vdash M \syn{\pi} S$
successfully synthesises the~type $S$, the~interpreter overrides its state
context $\Gamma$ with $\Gamma, x \is{\pi} S$ and any occurrence of $x$ in
a~subsequent term will be evaluated to $M$.

Third, the~\emph{eval} statement is evaluated and the~result printed out without
any modifications to the~interpreter state. The~eval statement is a semiring
element followed by a~type-synthesising term, so with the~statement
\begin{center}
  \janus{\quant{$\pi$} $M$}
\end{center}
interpreter synthesises the~type in its context $\Gamma$ using the~judgment
$\Gamma \vdash M \syn{\pi} S$ and then evaluates $M$.

\sectionwithtoc{De Bruijn index}

In the~Janus interpreter, functions are represented using
\emph{De Bruijn indices}, which simplifies the~implementation of substitution.
In \autoref{cha:lambdacalculus}, \lts used the~same set
$\mathbf{v}_0, \mathbf{v}_1, \mathbf{v}_2, \dots$ for both free and bound
variables. This caused a~risk of name collision during substitution, which we
had to avoid by replacing our bound variables with fresh ones. Instead, we can
represent every bound variable with a~natural number $n$ if there are $n-1$
$\lambda$s between the~bound variable occurrence and its binding $\lambda$.
The~binding variables disappear completely from the~syntax. De Bruijn indices
provide a~canonical description of every \lt; whereas, when using names for
local variables, \lt belongs to an~equivalence class identified by
the~equivalence $\equiv_\alpha$.

Some examples of \lts written using De Bruijn indices are:
\begin{align*}
  \lam x x \quad &\text{corresponds to} \quad \lambda. \; 0 \\
  \lam x {\lam y x} \quad &\text{corresponds to} \quad \lambda. \lambda. \; 1 \\
  \lam x {\lam y {\lam z {x z (y z)}}} \quad &\text{corresponds to} \quad
    \lambda. \lambda. \lambda. \; 2 \; 0 \; (1 \; 0) \\
  \lam z {(\lam y {y \; (\lam x x)})\;(\lam x {x z})}\quad&\text{corresponds to}
    \quad \lambda. (\lambda. \; 0 \; (\lambda. \; 0)) \; (\lambda. \; 0 \; 1).
\end{align*}

A~formal definition of terms with De Bruin indices and substitution on these
terms is described, for example, by \citet[Chapter~6]{pierce_2002}.

When printing a~term, our interpreter synthesises fresh names for variables
represented by De Bruin indices. Since the~implementation does not remember how
has the~user called a~variable which has been converted to a~De Bruin index,
the~name may differ when that variable is part of the~output. For example, if
the~user inputs term:
\begin{center}
  \janus{(\backslash{}a b c. a) : forall (\_ : U) (\_ : U) (\_ : U) . U}
\end{center}
then the~interpreter outputs equivalent term, except with variables
\janus{a}, \janus{b}, and \janus{c} replaced with
\janus{x}, \janus{y}, and \janus{z}:

\begin{center}
  \janus{\vmany{} (\vlam{x y z}{x}) : ∀ (\vmany{} \textbf{x} : \vuniv)\
    (\vmany{} \textbf{y} : \vuniv) (\vmany{} \textbf{z} : \vuniv) . \vuniv}
\end{center}

\sectionwithtoc{Program architecture}\label{sec:architecture}

This section describes the~implementation of our Janus interpreter. It is
a~console application written in Haskell~\citep{haskell_2010}. We have started
with the~code from the~\emph{LambdaPi} interpreter developed by
\citet{loh_et_al_2010}, but most of it was rewritten. Our implementation splits
up into eight Haskell modules:
\begin{itemize}
  \item the~\texttt{Janus.Evaluation} module implements the~β-reduction of
    terms;
  \item the~\texttt{Janus.Infer} module implements the~typing algorithm;
  \item the~\texttt{Janus.Judgment} module defines the~judgment monad;
  \item the~\texttt{Janus.Parsing} module implements the~parser of the~Janus
    language;
  \item the~\texttt{Janus.Pretty} module implements the~pretty printing;
  \item the~\texttt{Janus.REPL} module defines the~abstract REPL interface and
    one implementation of this interface, which provides an~interactive
    programming environment;
  \item the~\texttt{Janus.Semiring} module defines the~zero-one-many semiring;
  \item the~\texttt{Janus.Syntax} module defines the~abstract syntax tree type
    for the~representation of terms and implements substitution on this type.
\end{itemize}

The~\texttt{Janus.Syntax} module defines the~data types \texttt{CTerm} and
\texttt{ITerm}, representing the~type-checkable and type-synthesising terms
respectively. It also implements a~substitution on these terms, which replaces
a~De Bruijn index-based variable with a~type-synthesising term.

The~evaluation of terms is defined in the~\texttt{Janus.Evaluation} module. This
function evaluates everything as far as possible, including sub-terms; that is,
if, for example, λ-abstraction is not a~part of function application and cannot
be β-reduced, we still evaluate the~body of the~function.

Terms are evaluated into \emph{values} (the~data type \texttt{Value}), which are
represented using higher-order abstract syntax; meaning, some values are
represented by Haskell functions. This simplifies our implementation, because we
reuse the~function application of Haskell and do not have to define our
substitution on values. Unfortunately, Haskell functions cannot be shown or
compared for equality, which means that to print or equate values we have to
\emph{quote} them. Quoting converts the~values back into terms; it is done by
applying the~Haskell function to a~temporary variable, whose occurrences in
the~function body are then replaced with an~appropriate De Bruijn
index~\citep{loh_et_al_2010}.

The~typing algorithm runs in the~\texttt{Judgment} monad. Defined in
the~\texttt{Janus.Judgment} module, this monad holds the~context of typing
judgments in its state.

The~\texttt{Janus.Infer} module implements the~typing algorithm as described by
the~bidirectional judgments. The~main difference between the~typing rules and
the~code is the~lack of explicit handling of usage in the~rules, which we have
described in the~\nameref{sec:usage} section. The~function \texttt{synthType}
implements the~type-synthesising judgment and the~function \texttt{checkType}
the~type-checking judgment. Apart from returning the~usage, or a~usage and
the~synthesised type, typing algorithm can also fail; in this case the~result is
a~\texttt{TypingError}. In the~next section we look at examples of the~different
errors typing algorithm can report.

Parser for the~Janus language is implemented in the~\texttt{Janus.Parsing}
module. It is written using the~parser combinator library
\emph{Parsec}~[\citeauthor{parsec}]. Both the~function application and its
left-most sub-term are type-synthesising terms, which makes our grammar
left-recursive. Since a~recursive-descent parser can loop forever on such
grammars, Parsec provides the~\texttt{chainl1} combinator. Its type is
unfortunately too restrictive for our case, so our code implements analogous
functionality in the~\texttt{iTerm} function.

The~\texttt{Janus.Pretty} module implements pretty-printing for types that are
shown to the~user. Pretty-printer creates a~document by combining many smaller
parts and offers a~way to conveniently format and render it. We use
the~\emph{Prettyprinter} library~[\citeauthor{prettyprinter}]; a~modern,
well-documented option. Prettyprinter offers \emph{annotations} on the~document
which can determine how it should be rendered and we use them to colour
the~output. When pretty-printing the~terms (\texttt{ITerm} and \texttt{CTerm}),
we use the~\texttt{Printer} monad, which keeps track of variables occurring in
the~term and generates fresh names for the~De Bruijn index-based variables:
the~writer monad first collects all free variable names in the~term and then
the~reader monad tracks which names can be assigned to the~De Bruijn indices.

The~interactive programming environment is defined in the~\texttt{Janus.REPL}
module. It is implemented using
the~\emph{Repline}~[\citeauthor{repline}] library, which is a~wrapper around
\emph{Haskeline}~[\citeauthor{haskeline}]. We define
the~\texttt{MonadAbstractIO} type class to capture the~output capabilities of
the~interpreter. We can then have two different programming environments: one,
interactive, built on Haskeline; and another, scriptable, which we can use to
write tests of the~read-eval-print loop.

The~\texttt{Janus.Semiring} module defines \texttt{Semiring} type class and
the~one instance that is available in our interpreter, \texttt{ZeroOneMany}
semiring with the~$0, 1 \leq \omega$ (but not $0 \leq 1$) order.

Our implementation also has a~test suite. All tests are written using
the~\emph{Hspec} testing framework~[\citeauthor{hspec}]. We have three kinds of
tests: tests of the~parsing functions, in the~\texttt{ParseSpec} module; tests
of the~typing algorithm, in the~\texttt{TypeSpec} module; and integration tests
that simulate a~complete run of the~interpreter and evaluate several statements
in sequence, done using a~second instance of the~\texttt{MonadAbstractIO} class
in the~\texttt{IntegrationSpec} module.


